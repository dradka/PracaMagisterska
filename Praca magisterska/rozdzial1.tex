\chapter{Wstęp}
\section{Wprowadzenie}

\begin{spacing}{1.5}

Szybki rozwój medycyny (np. pojawianie się nowych leków, testów i procedur), rosnąca lawinowo ilość gromadzonych danych klinicznych, a także pojawianie się coraz większej liczby złożonych przypadków (co jest spowodowane m.in. starzeniem się społeczeństw) powoduje, że coraz bardziej istotne staje się wspomaganie lekarzy podczas podejmowania decyzji diagnostycznych i terapeutycznych. W tym celu tworzy się systemy wspomagania decyzji klinicznych (SWDK), przez które rozumie się wszelkie systemy komputerowe pomagające personelowi medycznemu podejmować decyzje\cite{Musen06}. Wśród SWD wyróżnia się: systemy do zarządzania informacją i wiedzą, systemy do zwracania uwagi, przypominania i alarmowania oraz systemy do opracowywania zaleceń.

DO SWDK z pierwszej grupy zalicza się wszelkie systemy służące do zbierania, przechowywania i udostępniania danych pacjentów. Przykładem takiego systemu jest Eskulap tworzony przez pracowników Politechniki Poznańskiej. Eskulap jest skierowany dla różnych placówek medycznych, m. in. szpitali, przychodni oraz aptek. Korzysta z niego wiele placówek na terenie całej Polski. Eskulap składa się z kilkudziesięciu modułów, m.in. eRejestracja, Apteka, Laboratorium, Elektroniczna Dokumentacja Medyczna. Do SWDK z drugiej grupy należą systemy wbudowywane w aparaturę pomiarową (np. monitorującą funkcje życiowe) lub laboratoryjną. Dzięki czemu na bieżąco można informować personel kliniczny o niewłaściwych (np. wykraczających poza normy) wartościach obserwowanych lub testowanych parametrów. Wreszcie do trzeciej grupy SWDK należą systemy, które dla konkretnego pacjenta przygotowują sugestie diagnostyczne i terapeutyczne, korzystając przy tym z szeroko rozumianych modeli decyzyjnych, które stosowane są do dostępnych danych pacjenta. 

Znanym przykładem diagnostycznego SWDK jest system Isabel \cite{Isabel}, który wykorzystuje model decyzyjny w formie odpowiednio poindeksowanych publikacji medycznych. Objawy i cechy charakterystyczne pacjenta są wprowadzane do systemu w formie tekstowej (mogą być także pobierane z elektronicznej karty pacjenta). System dopasowuje je do dostępnych publikacji i na podstawie tego dopasowania tworzy listę możliwych diagnoz dla pacjenta. Dodatkowo, istnieje możliwość przeglądana fragmentów publikacji, które związane są z poszczególnymi diagnozami. 

W praktyce większą popularność zyskują wytyczne postępowania klinicznego (ang. \textit{clinical practice guidelines}, CPG), które pozwalają opisać postępowanie dla pacjenta chorującego na określoną chorobę. Takie wytyczne są coraz częściej formalizowane oraz osadzane w SWDK w celu planowania i nadzorowania wykonywania terapii. Niestety, większość wytycznych jest opracowywana przy założeniu, że pacjent cierpi tylko na jedną przypadłość, co jest bardzo dużym ograniczeniem praktycznym. Z uwagi na proces starzenia się społeczeństw wzrasta liczba pacjentów, którzy cierpią jednocześnie na wiele schorzeń. W takich sytuacjach bezkrytyczne jednoczesne stosowanie wielu wytycznych może przynieść efekt odwrotny do zamierzonego, tzn. może pogorszyć jakość oferowanej opieki \cite{Boyd05}. Dlatego też niezwykle istotne jest szybkie wykrywanie możliwych konfliktów (niekorzystnych interakcji) pomiędzy wytycznymi i wprowadzenie takich zmian w wytycznych, aby uniknąć lub osłabić te konflikty. Wprowadzane zmiany mogą polegać na wprowadzeniu zamienników dla konfliktowych leków, przepisaniu dodatkowych leków, zmianie dawek leków lub też rezygnacji z części leków.

O znaczeniu problemu wykrywania i usuwania konfliktów między wytycznymi świadczy to, iż jest on jednym z „wielkich wyzwań” dla wspomagania decyzji klinicznych \cite{Sittig08}. Niniejsza praca jest próbą zmierzenia się z tym wyzwaniem poprzez opracowanie SWDK  wspomagającego jednoczesne stosowanie wielu wytycznych dla jednego pacjenta. 


\section{Cel i zakres pracy}

Cele główne pracy magisterskiej są następujące:
\begin{enumerate}
\item rozszerzenie podejścia do wykrywania i usuwania konfliktów wykorzystującego programowanie logiczne z ograniczeniami i opisanego w \cite{SzWilk2},
\item implementacja rozszerzonego podejścia w formie samodzielnego SWDK.
\end{enumerate}
Pierwszy z celów głównych wiąże się z następującymi celami szczegółowymi:
\begin{enumerate}
\item umożliwieniem stosowania więcej niż dwóch wytycznych, 
\item dopuszczeniem stosowania wielu zmian w wytycznych (wielu operacji modyfikujących wytyczne), 
\item uwzględnieniem stosowania dawek zarówno przy wykrywaniu konfliktów, jak i wprowadzania zmian w wytycznych.
\end{enumerate}
Drugi z celów głównych jest zdekomponowany na następujące cele szczegółowe:
\begin{enumerate}
\item opracowanie reprezentacji dla wytycznych, opisu konfliktów i wprowadzanych zmian,
\item uwzględnienie dodatkowych danych pacjenta, które nie występują w wytycznych, a które należy uwzględnić podczas wykrywania konfliktów,
\item stworzenie SWD pozwalającego na krokowe wykonywanie wytycznych, wyszukiwanie konfliktów oraz wprowadzanie zmian w wytycznych. System ten ma zostać zaimplementowany w języku Java oraz ma korzystać z dodatkowych bibliotek dostępnych na licencji \textit{open source}.
\end{enumerate}


\section{Struktura pracy}

W rozdziale 2 krótko omówiono wytyczne postępowania klinicznego oraz ich rolę w medycynie. W tym rozdziale zaprezentowano także podejścia do wykrywania i usuwania interakcji w wytycznych postępowania klinicznego. Rozdział 3 opisuje paradygmat programowania logicznego z ograniczeniami. Opisano w tym rozdziale podstawowe właściwości podejścia, a także zamieszczono przykład ilustrujący jego wykorzystanie. W rozdziale 4 zaprezentowano narzędzia i biblioteki użyte podczas wykonywania pracy magisterskiej. Rozdział 5 opisuje główną część pracy, czyli rozszerzenie podejścia do wykrywania i usuwania konfliktów oraz implementację systemu. W tym rozdziale przedstawiono poszczególne części systemu. Rozdział 6 prezentuje działanie systemu na wybranych scenariuszach klinicznych. Rozdział 7 stanowi wreszcie podsumowanie pracy. 
