\chapter{Wstęp}
\section{Wprowadzenie}

\begin{spacing}{1.5}
Medycyna jest ważnym działem nauki, ponieważ dotyczy każdego człowieka. 
% SW: Powiedziałbym raczej, że medycyna stara się leczyć (nie wszystkie choroby są niestety wylaczalne). Poza tym dobór odpowiednich leków jest elementem terapii.
Stara się leczyć ludzi z chorób przy zastosowaniu odpowiednich terapii, na które składają się m.in. prawidłowo dobierane leki. Istotne znaczenie przy leczeniu pacjenta ma wiedza, jaką dysponuje lekarz. Informatyka, która jest dziedziną nauki zajmującą się przetwarzaniem informacji, może pomóc lekarzom w zdobywaniu wiedzy. 
Ważnym elementem zdobywania wiedzy medycznej jest system wspomagania decyzji klinicznych. Jest to program komputerowy, który pozwala personelowi medycznemu podejmować odpowiednie decyzje dotyczące pacjenta. Wśród systemów tego typu wyróżnia się: 
systemy do zarządzania informacją i wiedzą, systemy do zwracania uwagi, przypominania i alarmowania oraz systemy do opracowywania zaleceń.
% SW: Tutaj mógłby Pan wprowadzić pojęcie systemów wspomagania decyzji klinicznych (SWDK), które wspomagają klinicystów w podejmowaniu różnego typu decyzji. Wśród tego typu systemów wyróżnia się (1) systemy do zarządzania danymi i wiedzą, (2) systemy do przypominania i ostrzegani oraz (3) systemy do wypracowywania sugestii (diagnostycznych i terapeutycznych) dla konkretnego pacjenta. 
% Podając definicję SWDK może Pan zacytować tę pracę:
% Musen, M. , Shahar, Y. , Shortliffe, E. H. (2006). Clinical decision support systems. W: Shortliffe, E.H., Cimino, J. (red.): Biomedical Informatics. Computer Applications in Health Care and Biomedicine, Springer, 698-736.
Dzięki tego typu systemom korzystającym z baz danych zawierających opisy leków, chorób, procedur medycznych oraz danych dotyczących pacjentów można w łatwiejszy sposób dobrać odpowiednią terapię dla pacjenta, co ma istotny wpływ na przebieg jego leczenia. 
% SW: Poniższe przykłady systemów nie do końca pasują do tego, co opisał Pan powyżej. Skupiłbym się na Eskulapie i ewentualnie podał przykłady systemów bardziej zaawansowanych (np. takich jak ISABEL -- http://www.isabelhealthcare.com), które podają m.in. sugestie diagnostyczne. Zrezygnowałbym też z systemu InfoMedica, który skupia się na części "szarej" (administracyjnej).

Przykładem takiego systemu medycznego jest Eskulap. Jest to system zrealizowany przez pracowników Politechniki Poznańskiej. Eskulap jest skierowany dla różnych placówek medycznych, m. in. szpitali, przychodni oraz aptek. Korzysta z niego wiele placówek na terenie całej Polski. Eskulap składa się z kilkudziesięciu modułów. Do modułów tych należą m. in. eRejestracja, Apteka, Laboratorium, Elektroniczna Dokumentacja Medyczna. Innym przykładem systemu medycznego jest Isabel. Jest to system diagnostyczny oparty na technologii webowej. Objawy i cechy charakterystyczne pacjenta są wprowadzane do systemu w formie tekstowej lub są pobierane z elektronicznego rekordu pacjenta. System zwraca w wyniku listę możliwych diagnoz pacjenta. Dodatkowo, istnieje możliwość skorzystania z książek, artykułów itp. dla każdej postawionej diagnozy. 

Ostatnio coraz większą popularność zyskują wytyczne postępowania klinicznego (ang. clinical practice guidelines, CPG), które pozwalają opisać postępowanie dla pacjenta chorującego na określoną chorobę. Niestety, wytyczne te opisują jedynie jedną przypadłość, co może doprowadzić do konfliktów w przypadku pacjenta chorującego na kilka dolegliwości. W przypadku takich pacjentów konieczne jest dobranie takich terapii, które nie pozostają ze sobą w konflikcie.
% SW: Tutaj warto byłoby wyjaśnić, że w praktyce klinicznej coraz większą popularność zyskują wytyczne/zalecenia postępowania (ang. clinical practice guidelines, CPG), które precyzują postępowanie dla pacjenta ze specyficznym problemem. Takie wytyczne są coraz częściej formalizowane oraz osadzane w SWDK w celu planowania i nadzorowania wykonywania terapii. Niestety, większość wytycznych jest opracowywana przy założeniu, że pacjent cierpi tylko na jedną przypadłość, co jest zbyt mocnym założeniem w praktyce. W dużej mierze, z uwagi na starzenie się społeczeństw, wzrasta liczba pacjentów, którzy cierpią jednocześnie na wiele schorzeń. W takich sytuacjach bezkrytyczne jednoczesne stosowanie wielu wytycznych może przynieść efekt odwrotny do zamierzonego, tzn. może pogorszyć jakość oferowanej opieki.
% Tutaj może Pan zacytować następującą pracę:
% Boyd, C. M., Darer, J., Boult, C., Fried, L. P., Boult, L., & Wu, A. W. (2005). Clinical Practice Guidelines and Quality of Care for Older Patients. JAMA: The Journal of the American Medical Association, 294(6), 716.
% SW: Tutaj nie chodzi tylko o leki (chociaż interakcje między nimi to najczęściej spotykane konflikty) -- mogą również pojawić się "niezgodności" pomiędzy pomiędzy zabiegami lub innymi formami terapii (np. ćwiczenia, dieta).
Jeśli takie konflikty występują, należy znaleźć odpowiedni zamiennik lub przepisać dodatkowy lek. W niektórych przypadkach trzeba zrezygnować z leków, aby nie pogorszyć stanu zdrowia pacjenta. Zadaniem systemu, którego dotyczy niniejsza praca magisterska, jest znajdowanie konfliktów wynikających ze stosowania wielu terapii u jednego pacjenta. 
% SW: Nie jest jasne, co rozumie Pan pod pojęciem procedury -- czy jest to synonim wtycznych (tak wynika z kontekstu)? Jeśli tak, to proszę zdecydować się na jeden termin (wytyczne) i stosować go konsekwentnie w całym tekście. Jeśli nie, to proszę zdefiniować pojęcie procedury.
W przypadku znalezienia konfliktu, należy dokonać zmian w wytycznych tak, aby zminimalizować skutki uboczne. 


\section{Cel i zakres pracy}

% SW: Cel i zakres pracy powinien zostać zmodyfikowany i "wyczyszczony" -- obecnie cele pracy są przesłonięte przez szczegóły implementacyjne. Cele pracy mógłby sformułować Pan następująco (to również pozwoli Panu na lepszą reklamę tego, co Pan zrobił).
% Mamy dwa cele główne: (1) rozszerzenie naszego podejścia do wykrywania i usuwania konfliktów opisanego w pracy [Wilk, S., Michalowski, W., Michalowski, M., Farion, K., Hing, M.M., Mohapatra, S.: Mitigation of adverse interactions in pairs of clinical practice guidelines using constraint logic programming. J Biomed Inform 46 (2), 341-53 (2013) oraz (2) implementacja rozszerzonego podejścia. Pierwszy z celów wiąże się z następującymi celami szczegółowymi: (1) dopuszczepnie możliwości stosowania więcej niż dwóch wytycznych, (2) dopuszczenie możliwości wprowadzania wielu zmian do wytycznych, (3) uwzględnienie dawek leków, zarówno podczas wykrywania interakcji, jak i wprowadzania zmian do wytycznych. Drugi z celów głównych jest dekomponowany na następujące cele szczegółowe: (1) opracowanie reprezentacji dla wytycznych, opisu interakcji oraz wymaganych zmian, (2) uwzględnianie dodatkowych danych pacjenta, które nie występują jawnie w wytycznych, (3) implementacja rozszerzonego podejścia w formie systemu pozwalającego na krokowe wykonywanie algorytmów oraz analizę możliwych scenariuszy postępowania z pacjentem. Realizowany w ramach pracy system ma zostać zaimplementowany w języku Java oraz wykorzystywać dodatkowe komponenty dostępne na licencji open source.

% SW: Zaproponowany nowy opis celów zastąpiłby poniższy tekst.
Celem pracy magisterskiej jest rozszerzenie podejścia opisanego w pracy "Mitigation of adverse interactions in pairs of clinical practice guidelines using constraint logic programming"\cite{SzWilk2}, a także implementacja rozszerzonego podejścia. Rozszerzenie podejścia wiąże się z umożliwieniem stosowania więcej niż dwóch wytycznych, dopuszczeniem stosowania wielu zmian w wytycznych oraz uwzględnieniem stosowania dawek zarówno przy wykrywaniu konfliktów, jak i wprowadzania zmian w wytycznych. Na implementację systemu składa się opracowanie reprezentacji dla wytycznych, opisu konfliktów i wprowadzanych zmian, uwzględnienie dodatkowych danych pacjenta, które nie występują w wytycznych, a także implementacja rozszerzonego podejścia. Implementacja rozszerzonego podejścia polega na wykonaniu systemu pozwalającego na krokowe wykonywanie algorytmów i znajdowanie konfliktów oraz wprowadzanie ewentualnych zmian w wytycznych. Tworzony system ma zostać zaimplementowany w języku Java oraz ma korzystać z dodatkowych bibliotek dostępnych na licencji open source. System ten ma wykorzystywać programowanie logiczne z ograniczeniami.


% SW: Tutaj ma Pan bardziej ogólny opis, niż w pierwszym punkcie -- można go tam przenieść.


\section{Struktura pracy}

W rozdziale 2 zamieszczono przegląd literatury na temat wytycznych postępowania klinicznego oraz ich roli w medycynie. W tym rozdziale zaprezentowano także podejścia do wykrywania i usuwania interakcji w wytycznych postępowania klinicznego. Rozdział 3 opisuje paradygmat programowania logicznego z ograniczeniami. Opisano w tym rozdziale podstawowe właściwości podejścia, a także zamieszczono przykład ilustrujący jego wykorzystanie. W rozdziale 4 zaprezentowano narzędzia i biblioteki użyte podczas wykonywania pracy magisterskiej. 
% SW: Biorąc pod uwagę bardziej precyzyjną specyfikację celów, powiedziałbym, że w rozdziale 5-tym opisuje Pan zarówno rozszerzenie metody, jak i implementację roszszerzenia.
Rozdział 5 opisuje główną część pracy, czyli rozszerzenie podejścia do wykrywania i usuwania konfliktów oraz implementację systemu. W tym rozdziale przedstawiono poszczególne części systemu. Rozdział 6 prezentuje przykłady działania systemu. Rozdział 7 stanowi podsumowanie pracy. 
