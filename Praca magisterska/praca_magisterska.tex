\documentclass[oneside,a4paper]{book}

\usepackage[pdftex]{graphicx}
\usepackage{amsmath}
\usepackage{amssymb}
\usepackage{textcomp}
\usepackage[utf8]{inputenc}
\usepackage[polish]{babel}
\usepackage[T1]{fontenc}
\usepackage{array}
\usepackage{tabularx}
\usepackage{enumitem}
\usepackage[justification=centering]{caption}

\raggedbottom

% pakiet stosowany do url'i w bibliografii, zamienia odnośniki na ładnie sformatowane
\usepackage{url}
% pakiety służące do numerowania i tworzenia algorytmów
\usepackage{algorithmic}
\usepackage{algorithm}
% redefinicja etykiety nagłówkowej listy algorytmów, domyślna jest po angielsku
\renewcommand{\listalgorithmname}{Spis algorytmów}

% pakiet do wyliczania skali, przydatny przy dużych obrazkach
\usepackage{pgf}
% pakiet służący do automatycznego sortowania odnośników do bibliografii
\usepackage[sort]{natbib}
% tworzenie listingów
\usepackage{listings}
% tworzenie figur wewnątrz figur
\usepackage{subfig}
% do automatycznego skracania nazw rozdziałów i podrozdziałów używanych w nagłówkach strony by mieściły się w jednej linii
\usepackage[fit]{truncate}
% fancyhdr - ładne nagłówki, definicja wyglądu nagłówka, numery stron będą umieszczane w nagłówku po odpowiedniej stronie
\usepackage{fancyhdr}
\pagestyle{fancy}
\renewcommand{\chaptermark}[1]{\markboth{#1}{}}
\renewcommand{\sectionmark}[1]{\markright{\thesection\ #1}}
\fancyhf{}
\fancyhead[LE,RO]{\bfseries\thepage}
% tutaj ograniczamy szerokość pola w nagłówku zawierającego nazwę rozdziału/podrozdziału do 95% szerokości strony
% redefinicja sposobu prezentacji nazw domyślnie wypisywanych wielkimi literami (np. domyślnie w nagłówku Spis treści będzie miał postać SPIS TREŚCI)
% Uwaga! to może popsuć wielkie litery w ogóle! Jak coś nie działa należy usunąć \nouppercase{} z poniższych definicji
\fancyhead[LO]{\nouppercase{\bfseries{\truncate{.95\headwidth}{\rightmark}}}}
\fancyhead[RE]{\nouppercase{\bfseries{\truncate{.95\headwidth}{\leftmark}}}}
\renewcommand{\headrulewidth}{0.5pt}
\renewcommand{\footrulewidth}{0pt}

% definicja typu prostego wymagana przez pierwsze strony rozdziałów itp.
% powyższe reguły niestety tych stron nie dotyczą, gdyż Latex automatycznie przełącza je pomiędzy fancy a plain
% w tym wypadku eliminujemy nagłówki i stopki na stronach początkowych
\fancypagestyle{plain}{%
 \fancyhead{}
 \fancyfoot{}
 \renewcommand{\headrulewidth}{0pt}
 \renewcommand{\footrulewidth}{0pt}
}

%\parskip 0.5in

% makro umożliwiające otaczanie symboli okręgami
\usepackage{tikz}
% brak justowania tekstu (bazą okręgu będzie linia tekstu)
\newcommand*\mycirc[1]{%
  \begin{tikzpicture}
    \node[draw,circle,inner sep=1pt] {#1};
  \end{tikzpicture}}

% pionowe justowanie tekstu, środek okręgu pokrywa się ze środkiem tekstu
\newcommand*\mycircalign[1]{%
  \begin{tikzpicture}[baseline=(C.base)]
    \node[draw,circle,inner sep=1pt](C) {#1};
  \end{tikzpicture}}

% zmiana nazwy twierdzeń i lematów
\newtheorem{theorem}{Twierdzenie}[section]
\newtheorem{lemma}[theorem]{Lemat}

% tworzenie definicji dowodu
\newenvironment{proof}[1][Dowód]{\begin{trivlist}
\item[\hskip \labelsep {\bfseries #1}]}{\end{trivlist}}
% \newenvironment{definition}[1][Definicja]{\begin{trivlist}
% \item[\hskip \labelsep {\bfseries #1}]}{\end{trivlist}}
% \newenvironment{example}[1][Przykład]{\begin{trivlist}
% \item[\hskip \labelsep {\bfseries #1}]}{\end{trivlist}}
% \newenvironment{remark}[1][Uwaga]{\begin{trivlist}
% \item[\hskip \labelsep {\bfseries #1}]}{\end{trivlist}}

% definicja czarnego prostokąta zwyczajowo dodawanego na koniec dowodu
\newcommand{\qed}{\nobreak \ifvmode \relax \else
      \ifdim\lastskip<1.5em \hskip-\lastskip
      \hskip1.5em plus0em minus0.5em \fi \nobreak
      \vrule height0.75em width0.5em depth0.25em\fi}

% poniższymi instrukcjami można sterować co ma być numerowane a co nie i co ma być wyświetlane w spisie treści
% \setcounter{secnumdepth}{3}
% \setcounter{tocdepth}{5}

% definicja czcionki mniejszej niż tiny (domyślnie takiej małej nie ma)
\usepackage{lmodern}
\makeatletter
  \newcommand\tinyv{\@setfontsize\tinyv{4pt}{6}}
\makeatother

% definicja jeszcze mniejszej czcionki
\usepackage{lmodern}
\makeatletter
  \newcommand\tinyvv{\@setfontsize\tinyvv{3.5pt}{6}}
\makeatother

% pakiet do obsługi wielostronicowych tabel
\usepackage{longtable}
\setlength{\LTcapwidth}{\textwidth}

\usepackage[section] {placeins}
\usepackage{multirow}
\usepackage{slantsc}

% interlinia
\usepackage{setspace}

% korekta marginesów - domyślnie latex ma jakieś kosmiczne
\usepackage{anysize}
\marginsize{3.5cm}{2.5cm}{2.5cm}{2.5cm}
% po zmianie marginesów konieczne jest wymuszenie przeliczenia nagłówków
\fancyhfoffset[E,O]{0pt}

\usepackage{titlesec}
\titlespacing*{\chapter}{0pt}{-50pt}{20pt}
\titleformat{\chapter}[display]{\normalfont\huge\bfseries}{\chaptertitlename\ \thechapter}{20pt}{\huge}

\setlength{\LTleft}{0pt}
\setlist[itemize]{nolistsep}
\setlist[enumerate]{nolistsep}
\begin{document}

\begin{titlepage}
\begin{center}

% Nagłówek
\includegraphics[width=0.3\textwidth]{img/put_logo.png}\\[0.1in]
\Large{Instytut Informatyki}\\
\normalsize
Wydział Informatyki\\
Politechnika Poznańska\\
ul. Piotrowo 2, Poznań \\[2cm]

\Large{PRACA DYPLOMOWA MAGISTERSKA}\\[1cm]

% Title
\Large \textbf {System wspomagający jednoczesne stosowanie wielu wytycznych klinicznych dla jednego pacjenta}\\[2cm]
% Submitted by
\begin{table}[h]
\centering
\begin{tabular}{lr}\hline \\ 
Dariusz Radka & 100383 \\ \\ \hline 
\end{tabular}
\end{table}

\vspace{1cm}

\normalsize Promotor \\
\begin{table}[h]
\centering
\begin{tabular}{c}\hline \\
dr hab. inż. Szymon Wilk \\ \\ \hline 
\end{tabular}
\end{table}

\vspace{2cm}
Poznań, 2015

\vfill

\end{center}

\end{titlepage}

% sekcja wstępna książki, numerowana rzymskimi
\frontmatter
% generacja strony tytułowej załączonej wcześniej
% \maketitle
% spis treści
\tableofcontents

% dodatkowa strona z podaniem źródła finansowania itp.
% wstawienie pustego symbolu
\null
% i wypełnienie nim całej dostępnej strony
\vfill
% a na jej dole dodajemy właściwy tekst
% w tym przypadku z wyrównaniem do prawej strony

% właściwa część książki, numerowana arabskimi od 1
\mainmatter
\def\arraystretch{1.5}

\chapter{Wstęp}
\section{Wprowadzenie}

\begin{spacing}{1.5}

Szybki rozwój medycyny (np. pojawianie się nowych leków, testów i procedur), rosnąca lawinowo ilość gromadzonych danych klinicznych, a także pojawianie się coraz większej liczby złożonych przypadków (co jest spowodowane m.in. starzeniem się społeczeństw) powoduje, że coraz bardziej istotne staje się wspomaganie lekarzy podczas podejmowania decyzji diagnostycznych i terapeutycznych. W tym celu tworzy się systemy wspomagania decyzji klinicznych (SWDK), przez które rozumie się wszelkie systemy komputerowe pomagające personelowi medycznemu podejmować decyzje\cite{Musen06}. Wśród SWD wyróżnia się: systemy do zarządzania informacją i wiedzą, systemy do zwracania uwagi, przypominania i alarmowania oraz systemy do opracowywania zaleceń.

DO SWDK z pierwszej grupy zalicza się wszelkie systemy służące do zbierania, przechowywania i udostępniania danych pacjentów. Przykładem takiego systemu jest Eskulap tworzony przez pracowników Politechniki Poznańskiej. Eskulap jest skierowany dla różnych placówek medycznych, m. in. szpitali, przychodni oraz aptek. Korzysta z niego wiele placówek na terenie całej Polski. Eskulap składa się z kilkudziesięciu modułów, m.in. eRejestracja, Apteka, Laboratorium, Elektroniczna Dokumentacja Medyczna. Do SWDK z drugiej grupy należą systemy wbudowywane w aparaturę pomiarową (np. monitorującą funkcje życiowe) lub laboratoryjną. Dzięki czemu na bieżąco można informować personel kliniczny o niewłaściwych (np. wykraczających poza normy) wartościach obserwowanych lub testowanych parametrów. Wreszcie do trzeciej grupy SWDK należą systemy, które dla konkretnego pacjenta przygotowują sugestie diagnostyczne i terapeutyczne, korzystając przy tym z szeroko rozumianych modeli decyzyjnych, które stosowane są do dostępnych danych pacjenta. 

Znanym przykładem diagnostycznego SWDK jest system Isabel \cite{Isabel}, który wykorzystuje model decyzyjny w formie odpowiednio poindeksowanych publikacji medycznych. Objawy i cechy charakterystyczne pacjenta są wprowadzane do systemu w formie tekstowej (mogą być także pobierane z elektronicznej karty pacjenta). System dopasowuje je do dostępnych publikacji i na podstawie tego dopasowania tworzy listę możliwych diagnoz dla pacjenta. Dodatkowo, istnieje możliwość przeglądana fragmentów publikacji, które związane są z poszczególnymi diagnozami. 

W praktyce większą popularność zyskują wytyczne postępowania klinicznego (ang. \textit{clinical practice guidelines}, CPG), które pozwalają opisać postępowanie dla pacjenta chorującego na określoną chorobę. Takie wytyczne są coraz częściej formalizowane oraz osadzane w SWDK w celu planowania i nadzorowania wykonywania terapii. Niestety, większość wytycznych jest opracowywana przy założeniu, że pacjent cierpi tylko na jedną przypadłość, co jest bardzo dużym ograniczeniem praktycznym. Z uwagi na proces starzenia się społeczeństw wzrasta liczba pacjentów, którzy cierpią jednocześnie na wiele schorzeń. W takich sytuacjach bezkrytyczne jednoczesne stosowanie wielu wytycznych może przynieść efekt odwrotny do zamierzonego, tzn. może pogorszyć jakość oferowanej opieki \cite{Boyd05}. Dlatego też niezwykle istotne jest szybkie wykrywanie możliwych konfliktów (niekorzystnych interakcji) pomiędzy wytycznymi i wprowadzenie takich zmian w wytycznych, aby uniknąć lub osłabić te konflikty. Wprowadzane zmiany mogą polegać na wprowadzeniu zamienników dla konfliktowych leków, przepisaniu dodatkowych leków, zmianie dawek leków lub też rezygnacji z części leków.

O znaczeniu problemu wykrywania i usuwania konfliktów między wytycznymi świadczy to, iż jest on jednym z „wielkich wyzwań” dla wspomagania decyzji klinicznych \cite{Sittig08}. Niniejsza praca jest próbą zmierzenia się z tym wyzwaniem poprzez opracowanie SWDK  wspomagającego jednoczesne stosowanie wielu wytycznych dla jednego pacjenta. 


\section{Cel i zakres pracy}

Cele główne pracy magisterskiej są następujące:
\begin{enumerate}
\item rozszerzenie podejścia do wykrywania i usuwania konfliktów wykorzystującego programowanie logiczne z ograniczeniami i opisanego w \cite{SzWilk2},
\item implementacja rozszerzonego podejścia w formie samodzielnego SWDK.
\end{enumerate}
Pierwszy z celów głównych wiąże się z następującymi celami szczegółowymi:
\begin{enumerate}
\item umożliwieniem stosowania więcej niż dwóch wytycznych, 
\item dopuszczeniem stosowania wielu zmian w wytycznych (wielu operacji modyfikujących wytyczne), 
\item uwzględnieniem stosowania dawek zarówno przy wykrywaniu konfliktów, jak i wprowadzania zmian w wytycznych.
\end{enumerate}
Drugi z celów głównych jest zdekomponowany na następujące cele szczegółowe:
\begin{enumerate}
\item opracowanie reprezentacji dla wytycznych, opisu konfliktów i wprowadzanych zmian,
\item uwzględnienie dodatkowych danych pacjenta, które nie występują w wytycznych, a które należy uwzględnić podczas wykrywania konfliktów,
\item stworzenie SWD pozwalającego na krokowe wykonywanie wytycznych, wyszukiwanie konfliktów oraz wprowadzanie zmian w wytycznych. System ten ma zostać zaimplementowany w języku Java oraz ma korzystać z dodatkowych bibliotek dostępnych na licencji \textit{open source}.
\end{enumerate}


\section{Struktura pracy}

W rozdziale 2 krótko omówiono wytyczne postępowania klinicznego oraz ich rolę w medycynie. W tym rozdziale zaprezentowano także podejścia do wykrywania i usuwania interakcji w wytycznych postępowania klinicznego. Rozdział 3 opisuje paradygmat programowania logicznego z ograniczeniami. Opisano w tym rozdziale podstawowe właściwości podejścia, a także zamieszczono przykład ilustrujący jego wykorzystanie. W rozdziale 4 zaprezentowano narzędzia i biblioteki użyte podczas wykonywania pracy magisterskiej. Rozdział 5 opisuje główną część pracy, czyli rozszerzenie podejścia do wykrywania i usuwania konfliktów oraz implementację systemu. W tym rozdziale przedstawiono poszczególne części systemu. Rozdział 6 prezentuje działanie systemu na wybranych scenariuszach klinicznych. Rozdział 7 stanowi wreszcie podsumowanie pracy. 


\chapter{Przegląd literatury}
\section{Wytyczne postępowania klinicznego}

% SW: Poniższy opis jest mocno zagmatwany i nie do końca wiadomo, jaka jest jego główna myśl przewodnia. Opisując wytyczne powinien Pan uwzględnić nastepujące elementy:
% (1) definicja wytycznych (mniej lub bardziej sformalizowany opis postępowania z pacjentem ze specyficznym problemem) oraz ich cel (standaryzacja opieki, ograniczenie zmienności w postępowaniu),
% (2) sposoby reprezentowania wytycznych (dokumenty tekstowe, reprezetacje formalne, w tym grafowe),
% (3) implementacja wytycznych w formie systemów komputerowych (możliwość personalizacji wytycznych i wykorzystania dostępnych danych pacjenta)
% (4) konflikty/interakcje wynikające ze stosowania wielu wytycznych jako jeden z głównych problemów ograniczających praktyczne wykorzystanie wytycznych oraz jedno z głównych wyzwań dla SWDK 
% W tym rozdziale mógłby zamieścić Pan przykłady różnych reprezentacji wytycznych (np. w formie tekstu oraz w formie grafu)


Wytyczne postępowania klinicznego to przygotowany w sposób systematyczny zestaw zaleceń odnośnie postępowania (diagnozowania, planowania terapii, realizacji procesu terapeutycznego) w specyficznych warunkach, np. dla określonego schorzenia czy też urazu \cite{Latoszek-Berendsen} (dla uproszczenia, w dalszej części tekstu będzie mowa o specyficznych chorobach).

Pierwsze wytyczne kliniczne zaczęły pojawiać się ponad 30 lat temu. Ich celem było ograniczenie zmienności w postępowaniu, jego ujednolicenie oraz optymalizacja pod względem wykorzystywanych kosztów i innych zasobów. Początkowo wytyczne były przygotowywane dla personelu pomocniczego (np. dla pielęgniarek), dopiero potem do grona potencjalnych "użytkowników" włączono lekarzy (było to związane z obawami środowiska lekarskiego przed tzn. \textit{cookbook medicine}, czyli medycyną według przepisu) \cite{Latoszek-Berendsen}. Praktyczna akceptacja wytycznych jest wciąż ograniczona -- o przyczynach takiego stanu rzeczy będzie mowa w dalszej części rozdziału.

Do reprezentowania wytycznych wykorzystywane są dwa modele:
\begin{enumerate}
\item model tekstowy, w którym wytyczne są prezentowane w postaci dokumentu tekstowego. Aby ułatwić jego przeglądanie lub przeszukiwanie, tekst jest wzbogacany o dodatkowe znaczniki (standard GEM \cite{Karras00}) nadające semantykę wybranym jego fragmentom,
\item model sieci zadań (ang. \textit{task network model}), w którym wytyczne są reprezentowane jako graf skierowany. Wierzchołki w grafie odpowiadają podstawowym krokom w wytycznych -- zebraniu danych, podjęciu decyzji, czy też wykonaniu akcji klinicznej. Krawędzie (łuki) wskazują natomiast na zależności kolejnościowe między poszczególnymi krokami. Ten właśnie model stanowi podstawę wielu formalnych reprezentacji wytycznych (np. GLIF3, PROforma, GLARE, SAGE \cite{Peleg} -- przegląd w \cite{Peleg2013}).
\end{enumerate}

W ciągu ostatnich rośnie popularność reprezentacji wytycznych wykorzystujących sieci zadań ze względu na możliwość ich osadzenia w systemach komputerowych -- mówi się nawet o nowej klasie wytycznych, czyli o wytycznych interpretowanych komputerowo (ang. \textit{computer interpretable guidelines}, CIG). Dzięki temu można budować SWDK, które pozwalają na połączenie wytycznych z danymi dostępnymi w wersji elektronicznej automatyzując tym samym proces opracowywania sugestii diagnostycznych i terapeutycznych dla konkretnego pacjenta. Takie "skomputeryzowane" wytyczne są przedmiotem intensywnych badań, które można podzielić na następujące kategorie \cite{Peleg2013}:
\begin{enumerate}
\item modelowanie i reprezentacja wytycznych,
\item pozyskiwanie i definiowanie wytycznych,
\item integracja z systemami szpitalnymi (zwłaszcza z elektroniczną kartą pacjenta),
\item walidacja i weryfikacja wytycznych,
\item wykonywanie wytycznych,
\item obsługa sytuacji wyjątkowych,
\item utrzymanie i "pielęgnacja" wytycznych,
\item współdzielenie wytycznych.
\end{enumerate}

Niniejsza praca magisterska dotyczy obsługi sytuacji wyjątkowych związanych z konfliktami, które mogą pojawić się podczas jednoczesnego wykonania wielu wytycznych dla jednego pacjenta. Odpowiada ona w ten sposób na wyzwanie związane z jednym z głównych ograniczeń wytycznych, tzn. skupieniem się na jednym specyficznym problemie. Ograniczenie to jest jedną z istotnych przyczyn słabego praktycznego przyjęcia i wykorzystania wytycznych \cite{Peleg2013} i jest też jednym z głównych wyzwań dla SWDK \cite{Sittig08}.

\section{Wykrywanie i usuwanie konfliktów w wytycznych}

\subsection{Programowanie logiczne z ograniczeniami}

W pracy \cite{SzWilk2} przedstawiono podejście wykorzystujące programowanie logiczne z ograniczeniami (ang. \textit{constraint logic programming}, CLP -- szczegółowy opis w rozdziale \ref{sec:clp}) do wykrywania i usuwania konfliktów dla pary wytycznych. Podejście to zakłada, że wytyczne prezentowane są w postaci grafu akcji (ang. \textit{actionable graph}). Graf akcji jest uproszczoną siecią zadań -- jest grafem skierowanym, w którym występują trzy typy węzłów:
\begin{itemize}
\item \textit{węzeł kontekstu} wskazujący na chorobę, której dotyczą wytyczne,
\item \textit{węzeł akcji} opisujący akcję kliniczną, jaką należy wykonać,
\item \textit{węzeł decyzyjny} opisujący decyzję i możliwe opcje.
\end{itemize}

W grafie akcji zrezygnowano z jawnego węzła odpowiadającego pozyskaniu danych i założono, że dane są pozyskiwane w węzłach decyzyjnych (np. poprzez zadanie odpowiedniego pytania lekarzowi). Przykład grafu akcji (rozszerzonego o możliwość zawierania ścieżek równoległych) przedstawiono na rys. \ref{fig:sciezki_rownolegle}.

Grafy akcji są automatycznie tłumaczone na modele logiczne i automatycznie przetwarzane. Podejście to wykorzystuje dodatkową wiedzę dziedzinową (nie pojawiającą się jawnie w wytycznych) reprezentowaną w formie operatorów interakcji (ang. \textit{interaction operators}) oraz modyfikacji (ang. \textit{revision operators}). Operator interakcji reprezentuje możliwy konflikt (zazwyczaj lek-lek lub lek-choroba), natomiast operator modyfikacji opisuje natomiast zmiany, jakie należy wprowadzić do modeli logicznych, aby konflikt usunąć. Oba operatory przedstawione są w formie wyrażeń logicznych.

Schemat działania podejścia przedstawiono na rys. \ref{fig:algorytm-clp}. Obejmuje on dwie fazy:
\begin{enumerate}
\item wyszukiwanie bezpośrednich konfliktów między wytycznymi (tzn. konfliktów, gdzie jeden wytyczne zalecają akcję \textit{A}, a drugie zabraniają tej akcji -- konflikty takie objawiają się jako niespójne wartości zmiennych współdzielonych przez modele logiczne) i ich usunięcie za pomocą operatorów modyfikacji,
\item wyszukanie pośrednich konfliktów (opisanych za pomocą operatorów interakcji) i ich usuniecie za pomocą operatorów modyfikacji.
\end{enumerate}

\begin{figure}
\begin{center}
\includegraphics[scale=0.6]{img/algorytm-clp.png}
\end{center}
\caption{Schemat działania podejścia wykorzystującego CLP \cite{SzWilk2}}
\label{fig:algorytm-clp}
\end{figure}

Podczas obu faz działania na podstawie modeli logicznych tworzone są programy CLP, które są następnie wykonywane/rozwiązywane. Uzyskane rozwiązanie wskazuje na terapię rozumianą jako ścieżki w grafach akcji, jakie należy przejść podczas leczenia pacjenta. Rozwiązanie uzyskane w ostatniej fazie jest ostateczną terapią prezentowaną lekarzowi.

Opisane podejście posiada kilka wad -- jest ograniczone tylko do dwóch wytycznych, nie pozwala na uwzględnianie dawek leków oraz pozwala tylko na proste zmiany (jedna operacja) wprowadzane przez operatory modyfikacji. Rozszerzenia zaproponowane w ramach tej pracy magisterskiej usuwają te wady.



\subsection{Logika pierwszego rzędu}

W pracy \cite{SzWilk} przedstawiono rozwinięcie opisanego powyżej podejścia, w którym CLP zastąpiono przez logikę pierwszego rzędu (ang. \textit{first-order logic}, FOL). Dzięki temu uzyskano znacznie bogatszą semantykę uwzględniającą zależności kolejnościowe między krokami, dawki leków, czy też złożone operacje modyfikacji wytycznych. Zmodyfikowano również algorytm wykrywania i usuwania konfliktów, aby uwzględniał wiele wytycznych. Podobnie jak poprzednio, wytyczne są podane w formie grafów akcji, a wiedza dziedzinowa w formie operatorów interakcji i rewizji.

Minusem tego podejścia są bardziej złożone modele reprezentujące wytyczne (konieczne jest definiowanie reguł kontrolujących procesem wnioskowania) oraz konieczność stosowania bardziej złożonych narzędzi (m.in. systemów do dowodzenia twierdzeń). Z uwagi na te komplikacje, w pracy magisterskiej skupiono się na wcześniejszym podejściu z CLP.



\subsection{Podejście wykorzystujące paradygmat mieszanej inicjatywy}

Paradygmat mieszanej inicjatywy (ang. \textit{mixed initiative paradigm}) polega na tym, że podczas rozwiązywania problemu użytkownik współpracuje z systemem komputerowym, a zatem nie ma tutaj działania w pełni automatycznego. Paradygmat ten został wykorzystany w pracy \cite{Piovesan} do wykrywania i usuwania konfliktów między wytycznymi. Podobnie jak we wcześniej opisywanych podejściach, również tutaj wytyczne reprezentowane są w postaci grafu (formalizm GLARE) składającego się z węzłów będących akcjami (dopuszcza sie zarówno akcje atomowe, jak i złożone, czyli plany) i krawędzi modelujących relacje między akcjami. Natomiast wiedza dziedzinowa o możliwych interakcjach reprezentowana jest w formie ontologii (wraz z towarzyszącą bazą wiedzy), która wykorzystuje istniejące terminologie i klasyfikacje medyczne, np. SNOMED CT dla pojęć medycznych i ACT dla leków.
  
Opisywane podejście stosuje dwie grupy metod do unikania i rozwiązywania zidentyfikowanych konfliktów:
\begin{enumerate}
	\item unikanie konfliktów
	\begin{itemize}
		\item wybieranie bezpiecznej alternatywy, tzn. takiej ścieżki w grafie, w której konflikt nie występuje,
		\item czasowe unikanie konfliktu, np. poprzez odpowiednie planowanie działań,
	\end{itemize}
	\item naprawienie konfliktów
	\begin{itemize}
		\item modyfikacja dawek leków,
		\item monitorowanie efektów (tutaj dopuszcza się mniej poważne konflikty i na bieżąco monitoruje ich następstwa),
		\item osłabianie interakcji poprzez rozszerzanie zaleceń o dodatkowe akcje.
	\end{itemize}
\end{enumerate}

Do przetwarzania wytycznych wykorzystywane jest wsteczne przeszukiwanie grafów reprezentujących wytyczne (np. w celu szukania bezpiecznych ścieżek), planowanie bazujące na celach (np. w celu monitorowania efektów konfliktowych akcji) oraz wnioskowanie temporalne (w celu unikania konfliktów lub planowania razem wzmacniających się akcji). Podejście to uwzględnia również pozytywne interakcje i pozwala albo na takie planowanie akcji, aby dochodziło do pożądanego wzmacniania ich efektów oraz wykrywa dublujące się akcje (np. podane tego samego leku pojawiające się w kilku zaleceniach).

Jak już wspomniano, podejście to nie jest automatyczne -- lekarz wskazać na odpowiedni sposób postępowania, a system stara się go zrealizować. Poza tym podejście to wymaga rozbudowanej i szczegółowej wiedzy dziedzinowej  podanej w formie ontologii.





\chapter{Programowanie logiczne z ograniczeniami}


Programowanie logiczne z ograniczeniami (ang. \textit{constraint logic programming}, CLP)\cite{CLP} wykorzystuje programowania logicznego (ang. \textit{logic programming}, LP) do rozwiązywania problemów spełaniania ograniczeń (ang. \textit{constraint satisfaction problem}, CSP). 

Program CLP składa się z następujących elementów:
\begin{itemize}
\item{zbioru zmiennych o wartościach ze skończonych dziedzin,}
\item{zbioru ograniczeń narzuconych na zmienne (np. X + Y > 5).}
\end{itemize}

Rozwiązaniem programu CLP jest takie przyporządkowanie wartości do zmiennych, aby spełnione zostały wszystkie zdefiniowane ograniczenia.

Poniżej opisano zastosowanie CLP do rozwiązania zagadki logicznej SEND + MORE  = MONEY \cite{Eclipse} pokazanej na rys. \ref{fig:sendmoremoney}. Rozwiązaniem tej zagadki jest takie przypisaniu cyfr z zakresu od 0 do 9 do zmiennych odpowiadających literom zawartym w równaniu, aby było ono spełnione. Każda zmienna (litera) powinna mieć unikalną wartość. Ponadto wartości zmiennych S i M muszą być różne od 0.

\begin{figure}
\begin{center}
\begin{verbatim}
    SEND
  + MORE
 -------
   MONEY
\end{verbatim}
\end{center}
\caption{Zagadka SEND + MORE = MONEY}
\label{fig:sendmoremoney}
\end{figure}
% SW: Tutaj warto wyjaśnić, że mówimy o środowisku do uruchamiania programów CLP, a nie o popularnym środowisku programistycznym. Poza tym warto dodać, że nieco bardziej szczegółowy opis środowiska znajduje się w punkcie 4.1.

Do rozwiązania tej zagadki służy następujący program pokazany na rys. \ref{fig:eclipse_program} i przygotowany w środowisku w ECLiPSe (chodzi tutaj o specjalizowane narzędzie dla CLP, a nie popularne środowisko programistyczne dla języka Java -- bardziej szczegółowy opis tego środowiska znajduje się w rozdziale \ref{sec:eclipse}):

\begin{figure}
\begin{verbatim}
:-lib(ic).
sendmore(Digits):-
Digits = [S,E,N,D,M,O,R,Y],
Digits :: [0..9],
alldifferent(Digits),
S #\= 0,
M #\= 0,
1000*S + 100*E + 10*N + D
+ 1000*M + 100*O + 10*R + E
#= 10000*M + 1000*O + 100*N + 10*E + Y,
labeling(Digits).
\end{verbatim}
\caption{Program w środowisku ECLiPSe rozwiązujący zagadkę}
\label{fig:eclipse_program}
\end{figure}

Można zauważyć, że poszczególne klauzule programu w ECLiPSe składają się ze zdań zakończonych kropką, poszczególne fragmenty zdań są oddzielone od siebie przecinkami. Znak równości między zmiennymi lub wartościami liczbowymi to „\#=”, znak nierówności to „\#\textbackslash=”. Można stosować także operatory and i or i przypisywać ich wartość do zmiennych za pomocą znaku równości. 

Po skompilowaniu tego programu wystarczy wywołać funkcję \texttt{sendmore(Digits)}, aby otrzymać rozwiązanie zagadki. Rozwiązaniem jest następujące przypisanie cyfr do zmiennych: S=9, E=5, N=6, D=7, M=1, O=0, R=8, Y=2. 

CLP może być wykorzystywane nie tylko do rozwiązywania popularnych zagadek logicznych (np. osiem królowych), ale jest także stosowane w poważnych, rzeczywistych problemach, m.in. m.in. harmonogramowaniu pracy lakierni samochodowej czy projektowaniu inteligentnych systemów okablowania dla dużych budynków \cite{niederlinski}.



\chapter{Wykorzystane biblioteki i narzędzia}

\section{ECLiPSe}

% SW: Powiedziałbym, że ECLiPSe to środowisko do wykonwania programów CLP (trudno powiedzieć, że służy do tworzenia aplikacji.
ECLiPSe\cite{EclipseSite} jest systemem typu open-source do wykonywania aplikacji wykorzystujących paradygmat programowania logicznego z ograniczeniami. System ten użyty został do testowania przykładowych procedur medycznych. 
% SW: Tutaj warto wyjaśnić, dlaczego nie zostało wykorzystane w systemie realizowanym w ramach pracy (np. z uwagi na ograniczaną możliwość intergracji z programami w języku Java).
Nie współpracuje natomiast z wykonanym w ramach pracy magisterskiej systemem, jest to odrębny program. 
% SW: Rozumiem, że tutaj pisze Pan o częściach głównego okna systemu ECLiPse -- sądzę, że należy to jasno powiedieć, a także zamieścić przykładowy ekran.
Okno systemu ECLiPSe składa się z trzech części. Pierwsza część służy do wprowadzania komend. Zamiast wprowadzania komend można wczytać gotowy program z pliku za pomocą polecenia Compile znajdującym się w menu File. Druga część programu wyświetla wyniki, trzecia natomiast pokazuje ewentualne błędy oraz inne komunikaty. Poniżej przedstawiono wygląd okna programu ECLiPSe.
\begin{figure}[H]
\centering
\includegraphics[width=0.7\textwidth]{img/okno.png}
\end{figure}


% SW: Taki fragmentaryczny opis składni jest mało przdatny. Napisałbym raczej, że przykładowy program przedstawiono w poprzedniej sekcji. Poza tym w poprzedniej sekcji umieściłbym tłumaczenia dotyczące interpretacji symboli "\#=" i "\#", ponieważ pojawiają się one w przykładowym programie.

%Komendy dla tego systemu składają się ze zdań zakończonych kropką, poszczególne fragmenty zdań są oddzielone od siebie przecinkami. Znak równości między zmiennymi lub wartościami liczbowymi to „\#=”, znak nierówności to „\#\textbackslash=”. Można stosować także operatory and i or i przypisywać ich wartość do zmiennych za pomocą znaku równości. 

\section{Choco 3}

% SW: Tutaj napisałbym, że Choco jest biblioteką w języku Java, która pozwala na rozwiązywanie problemów CLP. Oferuje zatem funkcjonalność zbliżoną do ECLiPSe'a, chociaż nie ma interfejsu użytkownika. Dostępna jest na licencji open source.

Choco\cite{Choco3} jest darmowym oprogramowaniem typu open-source, które pozwala na rozwiązywanie problemów CLP. Jest to biblioteka oparta o język Java w wersji 8. Główną klasą biblioteki jest klasa \texttt{Solver}. 
% SW: Tutaj duzym ułatwieniem byłoby wstawienie fragmetu kodu definiującego prosty problem i odwołanie się do poszczególnych linii w tłumaczeniu. Poza tym odpowiednie formatowanie tekstu (np. zastosowanie \texttt dla kodu) poprawi jego czytelność (zmieniłem formatowanie w tej sekcji). Poza tym zamiast pisać "metoda xxx w klasie Yyy" lepiej chyba zastosować konwencję "metoda Yyy.xxx".
Do obiektu typu \texttt{Solver} można dołączyć zmienną (klasa \texttt{IntVar}) podając obiekt \texttt{Solver-a} w ostatnim argumencie metody \texttt{VariableFactory.bounded}. Pozostałe argumenty tej metody to nazwa zmiennej oraz dolne i górne ograniczenie zmiennej. W pracy magisterskiej wykorzystywane są w większości zmienne, dla których dolne ograniczenie jest równe 0, a górne ograniczenie jest równe 1, czyli są to zmienne przyjmujące wartości prawda/fałsz. Za pomocą funkcji \texttt{Solver.post} można dodawać nowe ograniczenia. Ograniczenia tworzy się m.in. za pomocą klasy \texttt{IntConstraintFactory}. Jedną z podstawowych metod tworzących ograniczenia jest funkcja \texttt{arithm}. Przykładowo, można za jej pomocą określić, że suma dwóch zmiennych X i Y ma być mniejsza od 5. Po określeniu ograniczeń można uruchomić \texttt{Solver} i wygenerować rozwiązanie za pomocą metody \texttt{findSolution}. Kolejne rozwiązania można uzyskać za pomocą metody \texttt{nextSolution}. Odczytanie wartości zmiennej określonego rozwiązania polega na wywołaniu metody \texttt{IntVar.getValue}. 
Poniżej przedstawiono prosty program Choco3 szukający takich zmiennych X i Y (są to zmienne przyjmujące wartości 0 lub 1), których suma jest równa 1. Rozwiązaniem poniższego programu są dwa przypadki: X=1, Y=0 oraz X=0, Y=1.
\begin{verbatim}
Solver solver = new Solver("my first problem");
IntVar x = VariableFactory.bounded("X", 0, 1, solver);
IntVar y = VariableFactory.bounded("Y", 0, 1, solver);
solver.post(IntConstraintFactory.arithm(x, "+", y, "=", 1));
solver.findSolution();
do
{
   System.out.println("X="+x.getValue()+", Y="+y.getValue());
}while(solver.nextSolution());
\end{verbatim}


\section{Graphviz}

% SW: Graphviz jest pakietem do wizualizacji różnego typu grafów i zawiera kilka dedykowanych programów. My wykorzystujemy tylko jeden z nich - dot -- do tworzenia hierarchicznych grafów skierowanych.

% SW: Podobnie jak poprzednio, ten opis można skrócić i zastąpić częściowo przykładami prostych plików .dot wraz z wygenerowanymi obrazami. Warto też poprawić formatowanie. Wreszcie można zrezygnować z opisu parametrów wywołania programu -- ta informacja jest zbyt szczegółowa. Wreszcie mówiłbym raczej o formacie, a a nie rozszerzeniu .dot (to rozszerzenie wykorzystywane jest także przez inne programy, np. Word-a).

% SW: Podobnie jak poprzednmio, proszę podzielić ten tekst na krótsze akapity.

Graphviz\cite{Graphviz} jest oprogramowaniem służącym do wizualizacji grafów. Pozwala na konwersję pliku tekstowego w formacie dot do obrazu przedstawiającego graf. 
% SW: dot nie tylko automatycznie rozmieszcza węzły, ale również automatycznie prowadzi krawędzie, aby ograniczyć liczbę ich przecięć.
Program automatycznie porządkuje węzły na obrazie, nie jest konieczne podawanie pozycji węzłów, czyli ich współrzędnych. Ponadto, program automatycznie rysuje krawędzie tak, aby ograniczyć liczbę ich przecięć. Program z pakietu Graphviz o nazwie gvedit.exe jest programem okienkowym, który pozwala na wybranie w oknie dialogowym pliku o rozszerzeniu dot. Po wybraniu tego pliku albo wypisywana jest lista błędów, które należy poprawić, albo wyświetlany jest obraz przedstawiający graf. Podobną funkcjonalność ma program dot.exe, z tą różnicą, że jest to program konsolowy. Program dot.exe posiada 3 argumenty. Pierwszym argumentem jest ścieżka do pliku w formacie dot, drugim jest format generowanego obrazu (przykładowo dla uzyskania formatu png obrazka podajemy drugą wartość argumentu równą –Tpng). Między drugim a trzecim argumentem należy podać przełącznik „-o”. Trzecim argumentem jest ścieżka wynikowego obrazu. 

Jeśli chodzi o plik w formacie dot, jest to plik, który posiada swoją własną składnię. Na początku pliku umieszczone jest słowo „digraph”, po którym umieszcza się nazwę grafu. Wszystkie pozostałe właściwości grafu są umieszczone w bloku otoczonym nawiasami klamrowymi. W bloku tym można podać globalne atrybuty dla węzłów oraz krawędzi. Atrybuty dla węzłów mogą być podane po słowie node w bloku otoczonym nawiasami kwadratowymi, atrybuty są oddzielone od siebie przecinkami. Do przykładowych globalnych atrybutów węzłów należą m. in. kształt (box – prostokąt, circle – koło, diamond – romb), kolor wypełnienia, kolor konturu, grubość linii konturu, rodzaj czcionki, wielkość czcionki. Jeśli chodzi o globalne atrybuty krawędzi, to można je podać w podobny sposób jak globalne atrybuty węzłów, z tą różnicą, że zamiast słowa „node” należy podać słowo „edge”. Do atrybutów globalnych krawędzi należą przede wszystkim wielkość i rodzaj czcionki (krawędzie mogą posiadać etykiety). 

W następnym kroku można podać węzły i krawędzie z ich atrybutami. Atrybut pojedynczego węzła lub krawędzi, jeśli już wystąpił w globalnych atrybutach węzłów lub krawędzi, zostaje nadpisany. Opis pojedynczego węzła polega na podaniu jego unikalnego identyfikatora, a następnie jego atrybutów w bloku otoczonym nawiasami kwadratowymi (atrybuty są podawane po przecinku).  Krawędzie natomiast tworzy się, podając na początku identyfikator węzła źródłowego krawędzi, następnie należy umieścić tzw. strzałkę („->”), a na końcu identyfikator węzła docelowego. Po podaniu tych elementów można podać atrybuty krawędzi, przede wszystkim etykietę. Co ciekawe, krawędź może być także nieskierowana, wtedy zamiast strzałki („->”) należy umieścić podwójną kreskę („--”). 
Poniżej zaprezentowano bardzo prosty przykład pliku w formacie dot i jego graf:
\begin{verbatim}
digraph graf{
    node [shape=box, style=filled, fillcolor=green];
    A [label="Wierzchołek A"];
    B [label="Wierzchołek B"];
    C [label="Wierzchołek C"];
    A->B;
    A->C;
}
\end{verbatim}
\begin{figure}[H]
\includegraphics[width=0.3\textwidth]{img/graf.png}
\end{figure}

% SW: Poniższy fragment powinien przenieść Pan od opisu implementacji (nie dotyczy on bezpośrednio Graphviz-a).

%W systemie, którego dotyczy praca magisterska, wykorzystano program dot.exe. Za pomocą funkcji getRuntime klasy Runtime uzyskujemy instancję obiektu klasy Runtime. Na rzecz tego obiektu można następnie wywołać funkcję exec, której argumentem jest tablica łańcuchów znaków zawierająca w pierwszym elemencie ścieżkę do programu (w tym przypadku dot.exe), a w pozostałych elementach argumenty programu. Następnie należy wywołać funkcję waitFor dla obiektu klasy Process, który uzyskujemy w wyniku wywołania funkcji exec.

% SW: Najpierw powinien Pan opisać Graphviz-a (wraz z formatem .dot), a dopiero później bibliotekę służącą do odczytu plików .dot i tworzenia ich reprezentacji obiektowej w Javie.
\section{JPGD - A Java parser for Graphviz documents}

% SW: JPGD jest biblioteką ogólnego przeznaczenia i służy do analizy dowolnych grafów, a nie tylko tych opisujących wytyczne.

% SW: Poniższy opis wydaje się być zbyt szczegółowy, biorąc pod uwagę pomocniczy charakter biblioteki. Poza tym zastosowanie mają wszystkie te uwagi, które sformułowałem w przypadku opisu Choco 3 (dodanie przykładowego kodu źródłowego, zmiana formatowania i zapisu klas/metod).

% SW: Proszę podzielić ten tekst na akaipty -- obecnie mamy jeden wielki akapit obejmujący całą sekcję, a to utrudnia czytanie.

Biblioteka\cite{JPGD} ta służy do konwersji pliku o rozszerzeniu dot na obiekt klasy \texttt{Graph} posiadający listę obiektów klasy \texttt{Node} oraz \texttt{Edge}. Do konwersji wykorzystywany jest obiekt klasy \texttt{Parser}. Klasa \texttt{Parser} posiada funkcję \texttt{parse}, której konstruktor jako parametr przyjmuje obiekt klasy \texttt{FileReader} odwołujący się do określonego pliku o rozszerzeniu dot. W następnym kroku można odczytać obiekt klasy \texttt{Graph} z listy tych obiektów uzyskanej za pomocą funkcji \texttt{getGraphs} (jest to funkcja klasy \texttt{Parser}).

Węzły grafu można odczytać za pomocą funkcji \texttt{getNodes} wywołanej dla obiektu klasy \texttt{Graph}. Krawędzie grafu można natomiast uzyskać za pomocą funkcji \texttt{getEdges}, która również jest funkcją klasy \texttt{Graph}. Węzły oraz krawędzie posiadają atrybuty. Do atrybutów węzłów należy zaliczyć etykietę, kształt, kolor wypełnienia, kolor konturu i grubość linii konturu. Krawędzie posiadają przede wszystkim jeden istotny atrybut – etykietę. Odczytać wartości atrybutów można za pomocą funkcji \texttt{getAttribute}, której argumentem jest nazwa atrybutu. Ustawić wartości atrybutu można natomiast za pomocą metody \texttt{setAttribute}, której pierwszym argumentem jest nazwa atrybutu, a drugim jego wartość. 

Każdy węzeł grafu będącym w formacie dot posiada także swój unikalny identyfikator. Identyfikatory przechowywane są w obiektach klasy Id. Obiekt takiej klasy dla określonego węzła można uzyskać wywołując funkcję \texttt{getId} na rzecz obiektu klasy \texttt{Node}. Ponowne wywołanie funkcji \texttt{getId}, w tym przypadku dla obiektu klasy Id uzyskuje rzeczywisty identyfikator węzła typu \texttt{String}. 

Jeśli chodzi o krawędzie, to posiadają one możliwość odczytania węzła źródłowego oraz docelowego danej krawędzi. Jest to możliwe dzięki wywołaniu funkcji \texttt{getSource} (dla uzyskania węzła źródłowego) oraz \texttt{getTarget} (dla uzyskania węzła docelowego). Dzięki tym funkcjom uzyskujemy obiekt klasy \texttt{PortNode}, z którego następnie możemy uzyskać obiekt klasy \texttt{Node} za pomocą funkcji \texttt{getNode}. Ważną funkcją jest też funkcja \texttt{toString} wywoływana na rzecz obiektu klasy \texttt{Graph}. Pozwala ona na uzyskanie grafu w formacie dot zawierającym zmiany wprowadzone za pomocą metody \texttt{setAttribute} dla obiektów klasy \texttt{Node} lub \texttt{Edge}. 

Poniższy kod źródłowy prezentuje przykładowe wykorzystanie biblioteki JPGD do znalezienia krawędzi wyjściowych węzła n.
\newpage
\begin{verbatim}
public static ArrayList<Edge> getOutEdges(Graph graph, Node n)
{
    ArrayList<Edge> list = new ArrayList<Edge>();
    for(Edge e:graph.getEdges())
    {
       if(e.getSource().getNode()==n)
       {
           list.add(e);
       }
    }
    return list;
}
\end{verbatim}
% SW: Rozumiem, że musiał Pan samodzielnie poprawiać te błędy podczas realizacji pracy. Ten fragment warto przenieśc do podsumowania, gdzie opisuje Pan napotkane problemy.

\chapter{Implementacja systemu}

Rozdział ten opisuje implementację zrealizowanego w ramach pracy systemu. Jego działanie obejmuje następujące kroki: 
\begin{enumerate}
\item Na początku użytkownik wybiera choroby, na które choruje pacjent. 
\item Następnie program wyświetla graficzną reprezentację wytycznych dla wskazanych chorób oraz wyświetla listy pól wyboru, które pozwalają na udzielanie odpowiedzi na pytania zawarte w wytycznych. 
\item Po udzieleniu odpowiedzi na wybrane pytania (informacja nie musi być kompletna) program rozwiązuje problem CLP, w wyniku którego uzyskujemy listę konfliktów, które wystąpiły między wytycznymi oraz grafy wynikowe z wprowadzonymi zmianami. 
\item Po uzyskaniu rozwiązania problemu można wybrać inne odpowiedzi na pytania i wygenerować nowe rozwiązania problemu. Można także wybrać inne choroby i powiązane z nimi wytyczne.
\end{enumerate}
Poszczególne kroki zostały szczegółowo opisane w punkcie \ref{sec:kroki}. 

Program przetwarza wytyczne reprezentowane w postaci grafu skierowanego, w którym mogą wystąpić następujące rodzaje węzłów:
\begin{itemize}
\item \textit{węzeł początkowy} i \textit{końcowy} oznaczający odpowiednio rozpoczęcie i zakończenie wytycznych,
\item \textit{węzeł kontekstowy} opisujący chorobę, dla której sformułowane są wytyczne,
\item \textit{węzeł akcji} definiujący akcję (podanie leku, badanie, procedura), jaką należy wykonać,
\item \textit{węzeł decyzyjny} wskazujący na dane opisujące pacjenta, które należy pozyskać i sprawdzić w celu wyboru jednego z kilku możliwych sposobów postępowania,
\item \textit{węzeł równoległy} rozpoczynający lub kończący ścieżki równoległe.
\end{itemize}

Wszystkie węzły mają unikalne identyfikatory. Ponadto łuki (w dalszej części tekstu będziemy stosowali określenie krawędzie) wychodzące z węzłów decyzyjnych opisane są za pomocą etykiet odpowiadających poszczególnym decyzjom. Wreszcie węzły akcji i decyzyjne posiadają dodatkowe etykiety z dodatkowym, czytelnym ich opisem. Przykładowy graf przedstawiono na rys. \ref{fig:sciezki_rownolegle}. 

Grafy przechowywane są w plikach w formacie DOT. Format ten nie pozwala na użycie dodatkowej informacji semantycznej o typach węzłów (można określić jedynie ich identyfikatory oraz etykiety). Aby odróżnić węzły równoległe od decyzyjnych, sprawdzane są ich etykiety (a dokładnie ich dostępność lub brak). Poza tym, aby odróżnić węzeł rozpoczynający ścieżki równoległe od węzła kończącego, sprawdzane są liczby krawędzi wchodzących i wychodzących. Węzeł rozpoczynający ścieżki równoległe charakteryzuje się tym, że nie ma etykiety oraz posiada więcej niż jedną krawędź wyjściową. Natomiast węzeł kończący ścieżki równoległe nie posiada etykiety, ma więcej niż jedną krawędź wejściową oraz liczba jego krawędzi wyjściowych jest większa od zera (liściem jest zawsze węzeł końcowy).

\begin{figure}[H]
\centering
\includegraphics[width=0.7\textwidth]{img/asthma_sciezki_rownolegle.png}
\caption{Przykład ścieżek równoległych}
\label{fig:sciezki_rownolegle}
\end{figure}

W dalszych opisach wykorzystano pojęcia \textit{terapia} oraz \textit{element terapii}. \textit{Terapia} jest to pojedyncza ścieżka w grafie. \textit{Element terapii} natomiast to dla węzłów decyzji identyfikator węzła i etykieta wybranej krawędzi oddzielone znakiem zapytania, dla pozostałych węzłów \textit{elementem terapii} jest identyfikator węzła.

\section{Wykorzystywane dane}


Dane wykorzystywane oraz generowane przez program znajdują się w następujących katalogach:
\begin{itemize}
\item{\texttt{Algorytmy} – katalog zawiera pliki o rozszerzeniu DOT opisujące grafy reprezentujące dostępne wytyczne kliniczne,}
\item{\texttt{Konflikty} – katalog zawiera opisy konfliktów, jakie mogą wystąpić między wytycznymi oraz definicje zmiany, które należy wprowadzić w przypadku wystąpienia tych konfliktów,}
\item{\texttt{Grafy} – katalog zawiera zmodyfikowane grafy chorób przedstawiające aktualnie przebytą ścieżkę oraz grafy wynikowe prezentujące rozwiązania. Grafy są w dwóch formatach – tekstowym w formacie DOT oraz graficznym w formacie PNG. Podczas kończenia pracy programu zawartość tego katalogu jest kasowana}
\end{itemize}

\section{Główne klasy}
Program zaimplementowano w języku Java. Poniżej przedstawiono listę głównych klas wykorzystywanych w programie (i pojawiających się w dalszych opisach):
\begin{itemize}
\item{\texttt{AddToTherapy} - dodawanie identyfikatorów węzłów do listy opisującej konkretną terapię,}
\item{\texttt{ChocoClass} - rozwiązanie problemu CLP za pomocą solvera Choco,}
\item{\texttt{Color} - kolorowanie wierzchołków i krawędzi grafów,}
\item{\texttt{CreateTherapies} - generowanie terapii,}
\item{\texttt{ExecuteInteractions} - wprowadzanie zmian w terapiach w przypadku wykrycia konfliktów, }
\item{\texttt{GoForward} - przechodzenie do kolejnego węzła decyzyjnego, }
\item{\texttt{GraphFunctions} - przydatne metody związane z grafami, np. znalezienie węzłów docelowych określonego węzła,}
\item{\texttt{ImageGraph} - wyświetlanie grafów, }
\item{\texttt{MainClass} - obsługa przejścia między poszczególnymi krokami działania programu,}
\item{\texttt{RadioButtonList} - tworzenie i obsługa zdarzeń list pól wyboru służących do udzielania odpowiedzi na pytania,}
\item{\texttt{Results} - wyświetlanie wyników,}
\item{\texttt{Window} - główne okno programu.}
\end{itemize}

\section{Główne kroki działania}
\label{sec:kroki}

\subsection{Wybór chorób}
Celem tego kroku jest wybór tych wytycznych, które będą brane pod uwagę przy ustalaniu terapii. W katalogu \texttt{Algorytmy} program szuka plików posiadających rozszerzenie DOT. Dla każdego takiego pliku tworzone jest pole wyboru. Pole wyboru posiada etykietę równą nazwie choroby. Utworzone pole wyboru jest następnie dodawane do globalnej listy 
pól wyboru \texttt{Window.checkBoxGroup} oraz do panelu 
znajdującego się w lewym górnym rogu okna programu (rys. \ref{fig:wybor_chorob}). 
\begin{figure}[H]
\centering
\includegraphics[width=\textwidth]{img/wybor_chorob.png}
\caption{Panel wyboru chorób}
\label{fig:wybor_chorob}
\end{figure}
Po wybraniu chorób, tzn. po kliknięciu w odpowiednie pola wyboru i kliknięciu przycisku „Dalej”, nazwy wybranych chorób są dodawane do listy \texttt{MainClass.selectedDiseases} i program przechodzi do fazy wyświetlania grafów oraz udzielania odpowiedzi na pytania. 

\subsection{Wyświetlanie grafów oraz udzielanie odpowiedzi na pytania}
Ten krok pozwala na utworzenie graficznej reprezentacji wytycznych oraz wybór ścieżki w grafie (terapii) zgodnej z dostępnymi danymi pacjenta. Po wybraniu chorób i kliknięciu przycisku „Dalej” program dla każdej choroby odczytuje za pomocą metody \texttt{GraphFunctions.getGraph} grafy z plików w formacie DOT. Następnie program pobiera korzenie każdego z grafów za pomocą metody \texttt{GraphFunctions.get\-StartNode}, a potem wywołuje metodę \texttt{GoForward.goForward}, która przemieszcza się po grafie (startując w jego korzeniu) do momentu, gdy napotka pierwszy węzeł decyzji. 

Metoda \texttt{goForward} dodaje aktualny węzeł do listy elementów terapii. Następnie wykonuje pętlę \texttt{while}, której warunek kontynuacji obejmuje trzy przypadki. Pierwszy warunek sprawdza, czy węzeł posiada jedną krawędź wyjściową. Drugi warunek sprawdza, czy węzeł rozpoczyna ścieżki równoległe, a trzeci czy węzeł kończy ścieżki równoległe. 

Wewnątrz pętli wykonywane są następujące akcje. Po pierwsze wykonywana jest kolejna, wewnętrzna pętla \texttt{while} (jej warunek jest taki sam, jak pierwszy z warunków w pętli zewnętrznej), aby dodać do listy elementów terapii wszystkie węzły, które mają tylko jedną krawędź wyjściową, czyli droga w grafie, po której należy się poruszać jest jednoznacznie określona. Po wykonaniu tej pętli \texttt{goForward} zatrzymuje się na węźle, który jest liściem (nie posiada żadnej krawędzi wyjściowej), albo ma więcej niż jedną krawędź wyjściową.

Następnie następuje sprawdzenie, czy aktualny węzeł rozpoczyna ścieżki równoległe (jest to też drugi warunek w pętli zewnętrznej). Jeśli warunek ten jest spełniony, wywoływana jest metoda \texttt{parallel\-Path}. Metoda ta jest wywoływana również w dla trzeciego przypadku, czyli gdy uzyskany węzeł kończy ścieżki równoległe, a program nie przeszedł jeszcze przez wszystkie ścieżki równoległe. Metoda \texttt{GoForwared.parallel\-Path} jest w postaci pętli \texttt{while}, która działa dopóki program nie przejdzie przez wszystkie ścieżki równoległe związane z aktualnym węzłem i uzyskany węzeł nie jest węzłem decyzyjnym. Wszystkie przebyte po drodze węzły dodawane są do listy elementów terapii.  

Po wywołaniu metody \texttt{goForward} program wywołuje metodę \texttt{Color.color}, która zaznacza przebytą ścieżkę w grafie kolorując oraz pogrubiając kontury przebytych węzłów oraz przebyte krawędzie.
 
Po wywołaniu metody \texttt{color} wywołana zostaje metoda \texttt{ImageGraph.newImageGraph}, której zadaniem jest wygenerowanie i wyświetlenie nowego obrazu grafu. Na początku metoda zapisuje do pliku wynik metody \texttt{toString} wywołanej dla grafu. Następnie wywoływana jest metoda \texttt{ImageGraph.getImageGraphPath}, która uruchamia zewnętrzny program \texttt{dot} i tworzy z zapisanego wcześniej pliku tekstowego graf w postaci obrazu w formacie PNG. W kolejnym kroku metoda \texttt{ImageGraph.newImageGraph} tworzy obiekt klasy \texttt{BufferedImage} z wygenerowanym w poprzednim kroku obrazem. Później metoda dokonuje skalowania obrazu tak, aby mógł on się zmieścić w oknie (a dokładnie w przeznaczonym dla niego polu). 

Jeśli szerokość lub wysokość obrazu przekracza ustalony próg, obraz zmniejszany jest do 2/3 wielkości tak, aby był on czytelny (w tym przypadku do pola z obrazem dodawane są suwaki). Ponadto, jeżeli szerokość i wysokość obrazu jest mniejsza od wielkości pola, to na etykiecie umieszczany jest obraz bez skalowania (tzn. w skali 1:1).
 
Ostatnim krokiem jest wywołanie metody \texttt{RadioButtonList.createRadioButtonList}. Metoda ta dla każdego elementu terapii, który posiada znak zapytania tworzy panel. Elementy terapii zwierające znak zapytania odpowiadają krokom decyzyjnym. Pierwszym elementem panelu jest etykieta węzła. Pozostałe elementy stanowią pola wyboru z etykietami, których wartości są równe etykietom krawędzi wychodzących z węzła decyzyjnego. Do tych pól wyboru dodawany jest jeszcze jedno z etykietą „brak wartości”, przydatne w sytuacji, gdy dane nie są znane. Na końcu tworzony jest jeszcze jeden panel, tym razem dla pytania, na które jeszcze nie została udzielona odpowiedź -- dla niego zaznaczone jest pole wyboru z etykietą „brak wartości”. Przy pierwszym wyświetleniu grafu tworzony jest tylko ten panel. Ponadto, do każdego pola wyboru podczepiana jest metoda \texttt{RadioButtonList.updateRadioButtonList} obsługująca zdarzenia związane ze zmianą wartości pola.

Po wywołaniu metoda \texttt{updateRadioButtonList} na początku szuka elementu w liście elementów terapii, którego dotyczy wybrane pytanie (i związane z nim pole wyboru). Jeśli zaznaczone pole wyboru ma etykietę „brak wartości”, usuwane są wszystkie elementy terapii od elementu, którego dotyczy wybrane pytanie, do ostatniego elementu listy pytań. Innymi słowy następuje cofnięcie się w grafie, co oznacza, że pytania i odpowiedzi znajdujące się „poniżej” wybranej lokalizacji są odrzucane.

Jeśli natomiast zaznaczone pole wyboru nie posiada etykiety równej „brak wartości” i nie istnieje element na liście elementów terapii, który jest związany z pytaniem (użytkownik odpowiada na to pytanie po raz pierwszy), to do tej listy dodawany jest element o wartości równej \texttt{question?answer}, gdzie \texttt{answer} jest etykietą krawędzi, z którą związane jest zaznaczone pole wyboru.  Jeśli natomiast istnieje element związany z pytaniem (użytkownik modyfikuje wcześniej udzieloną odpowiedź), to w liście elementów terapii podmieniany jest element, który jest związany z pytaniem na wartość \texttt{question?answer}, a następnie usuwane są wszystkie elementy listy terapii, które się znajdują za podmienionym elementem (analogicznie jak w przypadku wyboru „braku wartości”). 

Następnie aktualnym węzłem staje się węzeł, do którego dochodzi krawędź związana z wybraną odpowiedzią (tzn. krawędź o etykiecie \texttt{answer} wychodzącej z węzła o identyfikatorze \texttt{question}). Węzeł ten jest punktem startowym dla kolejnego wywołania metody \texttt{goForward}. Metoda \texttt{goForward} nie jest wywoływana, gdy pole wyboru ma etykietę „brak wartości”. Potem metoda \texttt{updateRadio\-ButtonList} koloruje graf na nowo na podstawie zaktualizowanej listy elementów terapii. Tworzony jest także nowy obraz grafu za pomocą metody \texttt{newImageGraph}. Na końcu tworzona jest nowa lista pytań i odpowiedzi za pomocą metody \texttt{createRadioButtonList}. 

Zaktualizowane grafy prezentowane są po prawej stronie ekranu na osobnych zakładkach. Każda zakładka dotyczy wytycznych związanych z jedną z wybranych chorób. Z lewej strony ekranu pojawiają się natomiast zakładki z listami pytań i pól wyboru pozwalającymi na uzupełnienie danych pacjenta. W tym przypadku również jedna zakładka dotyczy jednej choroby. Przykładowy ekran z grafami i polami wyboru przedstawiono na rys. \ref{fig:wyswietlanie_grafow}.
\begin{figure}[H]
\centering
\includegraphics[width=\textwidth]{img/wyswietlanie_grafow.png}
\caption{Wyświetlanie grafów}
\label{fig:wyswietlanie_grafow}
\end{figure}

\subsection{Wyszukiwanie konfliktów}
Celem tego kroku jest znalezienie konfliktów pojawiających się między wytycznymi. Wyszukiwanie konfliktów rozpoczyna się od metody ChocoClass.solve. Na początku program szuka w katalogu \texttt{Konflikty} plików opisujących konflikty i sposoby ich rozwiązania, które można zastosować do aktualnego zestawu wytycznych. Nazwa każdego pliku opisującego konflikty składa się z listy nazw chorób oddzielonych przecinkami (konflikty pojawiają się między wytycznymi dla tych chorób). Jeśli jakakolwiek choroba z tej listy została wybrana podczas działania programu chorobach, plik zostaje użyty. 

Plik z konfliktami zawiera jeden lub więcej wpisów (każdy poświęcony jednemu konfliktowi), a wpis (umieszczony w jednej linii) składa się z dwóch części. Pierwsza część zawiera elementy, których jednoczesne wystąpienie powoduje konflikt. Elementy te są oddzielone spacjami. Druga część zawiera zmiany, jakie należy wprowadzić w przypadku zaistnienia konfliktu. Zmiany te są oddzielone od siebie przecinkami. Jeśli plik zostaje użyty, do listy \texttt{conflictsList} dodawane są konflikty, a do listy \texttt{interactionsList} zmiany. Ponadto, do listy \texttt{additionalQuestions} dodawane są te elementy opisujące konflikt, które oznaczają dodatkowe pytania (tzn. odwołują się do danych, które jawnie nie występują w wytycznych) -- nazwy tych elementów rozpoczynają się znakiem „\&”. Wreszcie zanegowane elementy konfliktów (rozpoczynające się od „not”) dodawane są do listy \texttt{notConflictElems}. Następnie program tworzy okienko dialogowe, które pozwala udzielić odpowiedzi na dodatkowe pytania. Pytania mogą być dwóch typów. Pierwszy typ występuje, gdy odpowiedni element konfliktu nie posiada znaku równości, mniejszości ani większości. Wtedy udzielana odpowiedź ma postać tak/nie. Drugi typ to „zmienna operator liczba”. Operator może być postaci „=”, „>”, „<”, „>=” lub „<=”. Dla tego typu elementu podawana jest wartość liczbowa w okienku dialogowym, a program sprawdza czy podana liczba spełnia warunek występujący w elemencie. 

Po udzieleniu odpowiedzi na wszystkie dodatkowe pytania program przechodzi do kolejnej części wyszukiwania konfliktów zrealizowanej w metodzie \texttt{ChocoClass.solveNextPart}. Metoda ta najpierw wywołuje metodę \texttt{ChocoClass.findSolutions}, która dla każdego możliwego konfliktu (odczytanego z pliku) wykonuje szereg operacji. Najpierw sprawdza, czy konflikt znajduje się na liście \texttt{foundConflicts}. Lista \texttt{foundConflicts} zawiera rzeczywiste konflikty, które zostały dotychczas zidentyfikowane przez program (tutaj należy zaznaczyć, że nie każdy możliwy konflikt musi zachodzić). Jeśli konflikt nie znajduje się na tej liście tworzony jest obiekt klasy \texttt{Solver}. Następnie dodawane są zmienne na podstawie wcześniej udzielonych odpowiedzi na dodatkowe pytania. Dokonuje tego metoda \texttt{ChocoClass.setAdditionalVariables}. Metoda ta sprawdza, czy pytanie jest typu tak/nie lub czy odpowiedzią na pytanie jest wartość liczbowa. W pierwszej sytuacji, jeśli odpowiedź jest równa tak, program tworzy zmienną typu \texttt{IntVar} o wartości równej jeden. Jeżeli odpowiedź jest równa nie, program tworzy zmienną \texttt{IntVar} o wartości równej zero. Jeśli odpowiedzią na pytanie jest wartość liczbowa, program tworzy zmienną \texttt{IntVar} o wartości równej podanej liczbie. 

Po wykonaniu metody \texttt{setAdditionalVariables} program wywołuje metodę \texttt{setVariables}. Metoda ta dla każdych wytycznych tworzy tablicę terapii. Dla każdej terapii (w ramach poszczególnych wytycznych) tworzona jest zmienna \texttt{IntVar} o nazwie \textit{<choroba>\_terapia<n>}, gdzie \textit{choroba} jest nazwą choroby, a \textit{n} jest indeksem terapii. Zmienna ta przyjmuje wartości zero, gdy  określona terapia nie zostaje użyta lub jeden, gdy zostaje użyta. Zmienna jest zapisywana w tablicy terapii. Następnie metoda tworzy listę \texttt{notConflictElemsTherapy}, do której dodawane są te elementy konfliktu z listy \texttt{notConflictElems}, które nie znajdują się na liście elementów konkretnej terapii, ale znajdują się w grafie związanym z terapią (innymi słowy są to te elementy, które występują w pozostałych terapiach dla danych wytycznych). 

Metoda \texttt{setVariables} tworzy też tablicę \texttt{vars}, która będzie zawierała zmienne wchodzące w skład pojedynczej terapii. Dla każdego elementu listy \texttt{notConflictElemsTherapy} w tablicy \texttt{vars} zapisywane są zmienne o nazwie \textit{not\_<X>}, gdzie X jest elementem terapii z \texttt{notConflictElems\-Therapy}. Zmienna ta przyjmuje wartość 0, gdy zmienna związana z elementem terapii jest równa 1 i odwrotnie. Ponadto, dla każdego elementu terapii zapisywana jest zmienna w tablicy \texttt{vars}. Jeśli zmienna odnosi się do elementu terapii (akcji) oznaczającego podanie leku, w ramach której określono jego dawkę, tworzona jest zmienna \textit{<X>\_dosage}, gdzie \textit{X} jest elementem terapii. Następnie metoda dodaje ograniczenie mówiące, że zmienna \textit{<choroba>\_terapia<n>} przyjmuje wartość jeden, tylko wtedy, gdy suma zmiennych należących do tablicy \texttt{vars} jest równa wielkości tej tablicy (innymi słowy, gdy udało się przejść przez całą ścieżkę odpowiadającą terapii i jednocześnie nie wystąpił żaden z zanegowanych elementów konfliktu). W przeciwnym razie \textit{<choroba>\_terapia<n>} przyjmuje wartość zero. Po przejściu przez wszystkie terapie dla danych wytycznych metoda dodaje ograniczenie polegające na tym, że suma zmiennych terapii choroby ma być równa jeden, czyli dla każdej choroby ma zostać użyta tylko jedna terapia. 

Po wykonaniu metody \texttt{setVariables} program dodaje ograniczenia odpowiadające konfliktom. Najpierw dodawane są ograniczenia dla tych możliwych konfliktów, których udało się uniknąć (tzn. po uwzględnieniu związanych z nimi ograniczeń udało się uzyskać poprawne rozwiązanie -- konflikty takie znajdują się na liście \texttt{avoidedConflicts}). Następnie w metodzie \texttt{ChocoClass.set\-ConflictConstraint} dodawane jest ograniczenie dla konfliktu sprawdzanego w aktualnej iteracji pętli. Najpierw tworzy ona listę \texttt{constraintsList} przechowującą to ograniczenie. Następnie metoda iteracyjnie przetwarza elementy wchodzące w skład aktualnego konfliktu:
\begin{itemize}
\item Jeśli element konfliktu jest zanegowany (jego opis zaczyna się od „not”), do listy \texttt{constraints\-List} dodawane jest ograniczenie \textit{not(<X> = 1)}, gdzie \textit{X} jest elementem terapii wymienionym w elemencie konfliktu. 
\item Jeśli element konfliktu zawiera jeden z operatorów „<”, „<=”, „=”, „>”, lub „>=” wówczas wywoływana jest metoda \texttt{ChocoClass.conflictWithDosage}, która sprawdza, jaki charakter ma element terapii z elementu konfliktu i dodaje odpowiednie ograniczenia do \texttt{constraints\-List}. Jeśli element terapii jest związany z dodatkowym pytaniem (jego nazwa rozpoczyna się od „\&”), wówczas dodawane jest ograniczenie \textit{<X> <operator> <wartość>}, gdzie \textit{X} jest elementem terapii, a \textit{wartość} jest liczbą występującą w elemencie konfliktu. W przeciwnym razie (element terapii jest związany z akcją, w tym z podaniem leku), do \texttt{constraintsList} dodawane są dwa ograniczenia. Pierwsze jest w postaci \textit{<X> = 1} (odpowiada ono wykonaniu wskazanej akcji), natomiast drugie ma formę \textit{<X>\_dosage <operator> <wartość>} i dotyczy dawki związanej z daną akcją. 
\item W pozostałych przypadkach do \texttt{constraintsList} dodawane jest ograniczenie \textit{<X> = 1}, gdzie \textit{X} to odpowiedni element terapii.
\end{itemize}

Po zakończeniu przetwarzania poszczególnych elementów konfliktu do obiektu klasy \texttt{Solver} dodawane jest ograniczenie w formie \textit{not(and(\texttt{constraintsList}))}, aby zabezpieczyć się przed wystąpieniem aktualnego konfliktu.

Następnie wywoływana jest metoda \texttt{findSolution}, która szuka rozwiązania problemu. Jeśli rozwiązanie istnieje, aktualny konflikt dodawany jest to listy do \texttt{avoided\-Conflicts}. Jeśli natomiast rozwiązanie nie istnieje, aktualny konflikt jest dodawany do \texttt{foundConflicts}, a do listy \texttt{interactionsList} dopisywane są zmiany powiązane z danym konfliktem. Ponadto, gdy nie ma rozwiązania program wywołuje metodę \texttt{ExecuteInteractions.execute\-Interactions}, która dokonuje zmian w wytycznych (a dokładnie w terapiach), a także program wywołuje rekurencyjnie metodę \texttt{findSolutions}, aby sprawdzić, czy wprowadzone zmiany nie spowodowały wystąpienia konfliktów, które zostały wcześniej sprawdzone, oraz aby sprawdzić kolejne konflikty. 

Ostatecznie, program znajduje rozwiązania problemu z ograniczeniami dla tych konfliktów, które znajdują się na liście \texttt{avoidedConflicts} (jak już wspomniano, są to konflikty, których udało się uniknąć -- dla konfliktów, które wystąpiły, wprowadzono odpowiednie zmiany do terapii). Po wygenerowaniu pierwszego rozwiązania program tworzy listę o nazwie \texttt{solutions}. Następnie w pętli, która działa dopóki istnieje kolejne rozwiązanie, program zapisuje do zmiennej \texttt{solution} nazwy zmiennych, które w rozwiązaniu posiadają wartość równą jeden. Następnie, jeśli zmienna \texttt{solution} nie znajduje się jeszcze w liście \texttt{solutions}, dodawana jest do tej listy. 

Na końcu program do listy \texttt{therapies} dodaje rozwiązania. Polega to na tym, że dla każdego rozwiązania z listy \texttt{solutions} znajdujemy odpowiadające mu terapie w liście \texttt{therapiesDiseases}. Lista \texttt{therapiesDiseases} zawiera terapie wszystkich wybranych chorób zgodne z udzielonymi odpowiedziami na pytania znajdujące się w wytycznych. Znalezienie odpowiedniej terapii polega na odczycie nazwy choroby i identyfikatora terapii ze zmiennych <choroba>\_terapia<n> znajdujących się w liście \texttt{solutions}.

\subsection{Wyświetlanie wyników}
\label{sect:revisions}
Ostatni krok polega na wyświetleniu grafów wynikowych prezentujących rozwiązania, a także utworzeniu listy znalezionych konfliktów wraz z wprowadzanymi zmianami. Program prezentuje wyniki za pomocą metody \texttt{Results.setResults}. Na początku metoda wywołuje inną metodę o nazwie \texttt{Results.setGraphs}. Zajmuje się ona modyfikowaniem grafów, wprowadzając niezbędne zmiany usuwające znalezione konflikty.
Dla każdej modyfikacji sprawdzany jest jej typ, który może być jednym z następujących:
\begin{itemize}
\item \textit{replace <X> with <Y>}, gdzie węzeł \textit{X} zamienia się na węzeł \textit{Y},
\item \textit{add <X> before/after <Y>}, gdzie węzeł \textit{X} jest dodawany przed lub po elemencie \textit{Y},
\item \textit{remove <X>}, węzeł \textit{X} jest usuwany,
\item \textit{increase\_dosage/decrease\_dosage <X> <DV>}, gdzie dawka leku z węzła \textit{X} jest zwiększana lub zmniejszana o wartość \textit{DV},
\item \textit{change\_dosage <X> <V>}, gdzie dawka leku z węzła \textit{X} jest ustalana na \textit{V}.  
\end{itemize}

Zmiana grafu uzależniona jest od typu modyfikacji i przeprowadzana jest w następujący sposób:
\begin{itemize}
\item Dla modyfikacji \textit{replace}, najpierw wyszukiwany jest węzeł \textit{X}. Po znalezieniu takiego węzła z pliku \texttt{Konflikty/nazwy.txt} odczytywana jest etykieta węzła \textit{Y}. Wreszcie identyfikator i etykieta znalezionego węzła są zmieniane na nowe wartości. 
\item Dla modyfikacji \textit{add}, poszukiwany jest węzeł \textit{Y}, przed lub za którym ma zostać umieszczony nowy węzeł \textit{X}. Następnie program tworzy węzeł \textit{X}, nadaje mu etykietę pobraną z pliku \texttt{nazwy.txt} i dodaje węzeł do grafu. Jeśli węzeł \textit{X} jest wstawiany po elemencie postaci \textit{pytanie?odpowiedź}, wówczas staje się on węzłem docelowym dla krawędzi z etykietą \textit{odpowiedź}, oraz wstawiana jest dodatkowa krawędź od węzła \textit{X} do poprzedniego węzła docelowego. W przeciwnym razie wstawiana jest krawędź z Y do X (dla \textit{add after}) lub z X do Y (dla \textit{add before}).
\item Dla modyfikacji \textit{remove} atrybuty usuwanego węzła \textit{X} modyfikowane są w taki sposób, że węzeł staje się niewidoczny na wynikowym grafie
\item Dla modyfikacji \textit{increase\_dosage}, \textit{decrease\_dosage} i \textit{change\_dosage} odpowiednio zmienia się końcową część etykiety zmienianego węzła \textit{X}, gdzie w nawiasach kwadratowych umieszczona jest zmieniona dawka leku związanego z węzłem. 

\end{itemize}

Na zakończenie metoda \texttt{setGraphs} dla każdego grafu wywołuje metodę \texttt{color} zaznaczającą przebyte węzły i krawędzie, a następnie metodę \texttt{newImageGraph}, która powoduje wygenerowanie grafu w postaci obrazkowej. 

Po wywołaniu metody \texttt{setGraphs} program tworzy także tekstową reprezentację uzyskanych rozwiązań. Dla każdej z otrzymanych terapii obejmuje ona identyfikatory oraz etykiety elementów terapii (czyli odwiedzonych węzłów w grafach). Reprezentacja ta zawiera również opis napotkanych konfliktów oraz listę wprowadzonych zmian. Przy ustalaniu etykiet węzłów grafach wykorzystywane są informacje z pliku \texttt{nazwy.txt}.


\section{Przykład działania programu}
W tym punkcie przedstawiono działanie programu na wybranym przykładzie klinicznym obejmującym wytyczne dla dwóch chorób: migotania przedsionków (ang. \textit{atrial fibrillation}, rys. \ref{fig:afib_przyklad}) oraz przewlekłej choroby nerek (ang. \textit{chronic kidney disease}, rys. \ref{fig:ckd_przyklad}). W przypadku węzłów odpowiadających akcjom i decyzjom podano ich etykiety oraz identyfikatory (w nawiasach okrągłych).

\begin{figure}[H]
\centering
\includegraphics[scale=0.5]{img/afib-ver-4_przyklad.png}
\caption{Wytyczne dla migotania przedsionków}
\label{fig:afib_przyklad}
\end{figure}
\begin{figure}[H]
\centering
\includegraphics[scale=0.4]{img/ckd-simplified-ver-5_przyklad.png}
\caption{Wytyczne dla przewlekłej choroby nerek}
\label{fig:ckd_przyklad}
\end{figure}

W przypadku wytycznych dla migotania przedsionków (rys. \ref{fig:afib_przyklad}) program zatrzymuje się na pierwszym węźle decyzyjnym \textit{AFib type?}, zapisując wcześniej do listy elementów terapii węzeł startowy, węzeł kontekstowy określający chorobę oraz węzeł \textit{Perform CHA2DS2-VASc test}. Po wskazaniu przez użytkownika odpowiedzi \textit{paroxysmal} program dodaje do listy \textit{d\_afib?paroxysmal}, a następnie trzy węzły: \textit{Short-term therapy: BB}, \textit{Rhythm control therapy: amiodarone} i \textit{Long-term oral anticoagulation: warfarin}. Następnie program dodaje węzeł rozpoczynający ścieżki równoległe, następnie dwa węzły znajdujące się na ścieżkach równoległych (najpierw węzeł \textit{CCB}, następnie węzeł \textit{ACE inhibitor}), a później węzeł kończący ścieżki równoległe. Ostatecznie program dodaje węzeł końcowy grafu. 

W wytycznych dla choroby nerek (rys. \ref{fig:ckd_przyklad}) program zatrzymuje się na węźle decyzyjnym \textit{GFR level?}, dodając po drodze do listy elementów terapii węzeł startowy oraz węzeł choroby. Po udzieleniu odpowiedzi \textit{<40} program dodaje do listy \textit{d\_gfr?<40}, a następnie trafia na węzeł rozpoczynający ścieżki równoległe, który również jest dodawany do listy. W kolejnym kroku program zatrzymuje się na  decyzji znajdującej się na lewej ścieżce równoległej: \textit{Anemia present?}. Po uzyskaniu odpowiedzi \textit{no} program dodaje do listy element \textit{d\_anemia\_present?no}, a następnie wraca do prawej ścieżki równoległej i zatrzymuje się na decyzji \textit{Mineral metabolism abnormalities?}. Po uzyskaniu odpowiedzi \textit{absent} program dodaje do listy \textit{d\_metabolism\_anomalies\_present?absent}. Następnie program umieszcza na liście węzeł kończący ścieżki równoległe, który jednocześnie rozpoczyna kolejne ścieżki równoległe. W kolejnym kroku program dodaje element znajdujący się na lewej ścieżce równoległej, czyli \textit{CV risk management}, a następnie elementy znajdujące się na prawej ścieżce równoległej, czyli \textit{Antiplatelets: aspirin} oraz \textit{ACE inhibitor}. Ostatecznie dodawany jest węzeł kończący ścieżki równoległe, węzeł \textit{Lifestyle management} oraz węzeł końcowy grafu.

Po określeniu terapii program przechodzi do fazy wyszukiwania konfliktów. Najpierw program odczytuje plik z konfliktami dotyczącymi migotania przedsionków, przewlekłej choroby nerek oraz nadciśnienia. Plik ten ma następującą zawartość:
\begin{verbatim}
c_htn c_ckd:remove a_step1_acei,remove a_step1_ccb
c_afib c_ckd c_htn:remove a_step3_diuretric
c_afib c_ckd:replace a_antiplatelets with a_warfarin,replace a_rct_a with a_bb
c_afib c_ckd &CHA2DS2-VASc>2:replace a_mt_asa with a_warfarin_2
c_afib c_ckd &CHA2DS2-VASc<=1:replace a_oa_w with a_aspirin_1,
replace a_ltoa_w with a_aspirin_2
\end{verbatim}
Następnie program prosi o uzupełnienie danych pacjenta poprzez podanie wartości zmiennej \textit{CHA2\-DS2-VASc}. Użytkownik uzupełnia dane -- niech wartość tej zmiennej będzie równa 5, a program zaczyna wyszukiwać konflikty. Ostatecznie program znajduje dwa konflikty: (\textit{c\_afib c\_ckd} -- współwystąpienie obu chorób) oraz (\textit{c\_afib c\_ckd \&CHA2DS2-VASc>2} -- podwyższona wartość CHA2DS2-VASc w połączeniu z migotaniem przedsionków). Pierwsze dwa konflikty nie wystąpiły, ponieważ pacjent nie choruje na nadciśnienie (\textit{c\_htn}), więc odpowiednie wytyczne nie są rozważane. Ostatni konflikt nie wystąpił natomiast dlatego, że zmienna CHA2DS2-VASc przyjmuje wartość większą od 1.

W kolejnym kroku program tworzy zmodyfikowane grafy wynikowe, które zawierają zmiany wprowadzone w celu uniknięcia konfliktów. Graf dla migotania przedsionków został przedstawiony na rys. \ref{fig:afib_rozw}, natomiast graf dla przewlekłej choroby nerek jest na rys. \ref{fig:ckd_rozw}. W grafie dla migotania przedsionków węzeł \textit{Maintenance therapy: aspirin} został zmieniony na \textit{Maintenance therapy: Warfarin} oraz węzeł \textit{Rhytm control therapy: amiodarone} na węzeł \textit{BB}. Natomiast W grafie dla przewlekłej choroby nerek węzeł \textit{Antiplatelets: aspirin} został zmieniony na \textit{Warfarin}.

\begin{figure}[H]
\centering
\includegraphics[scale=0.5]{img/rozwiazanie1afib-ver-4_przyklad.png}
\caption{Wytyczne dla migotania przedsionków}
\label{fig:afib_rozw}
\end{figure}
\begin{figure}[H]
\centering
\includegraphics[scale=0.4]{img/rozwiazanie1ckd-simplified-ver-5_przyklad.png}
\caption{Wytyczne dla przewlekłej choroby nerek}
\label{fig:ckd_rozw}
\end{figure}

\chapter{Działanie systemu}
W tym rozdziale zaprezentowane zostało działanie stworzonego systemu z wykorzystaniem wybranych przykładów wytycznych klinicznych (wszystkie wytyczne zostały skonsultowane z lekarzami, przy czym część została uproszczona na potrzeby wcześniejszych publikacji, np. \citep{SzWilk,SzWilk2}). Dla każdego przykładu przedstawione zostały grafy reprezentujące zastosowane wytyczne. Na grafach tych zaznaczono ścieżki, które zostały wybrane podczas etapu zbierania danych pacjenta i udzielania odpowiedzi na pytania związane z węzłami decyzyjnymi. Odpowiedzi na pytania są także przedstawione w formie tekstowej. Następnie, dla każdego przykładu przedstawiono listę możliwych konfliktów oraz zmiany, jakie należy wprowadzić w przypadku ich wystąpienia (do opisu zmian wykorzystano składnię wprowadzoną w rozdziale \ref{sect:revisions}. W celu zachowania większej zwięzłości prezentacji w opisach konfliktów i wymaganych zmian zastosowano identyfikatory węzłów -- są one przedstawione na grafach (w nawiasach po etykietach poszczególnych węzłów). Ponadto dla każdego przykładu pogrubioną czcionką zaznaczono znalezione konflikty (nie wszystkie możliwe konflikty musiały wystąpić). Na końcu każdego przykładu podane zostały grafy wynikowe. Dodane węzły w grafach wynikowych zostały oznaczone niebieską czcionką.

\section{Przykład 1 - atak astmy i wrzód trawienny}
Wytyczne dla ataku astmy (ang. \textit{asthma exacerbation}) przedstawiono na rys. \ref{fig:ag_ae}, natomiast wytyczne dla wrzodu trawiennego (ang. \textit{peptic ulcer}) przedstawiono na rys. \ref{fig:pu}.

\begin{figure}[H]
\centering
\includegraphics[scale=0.45]{img/asthma.png}
\caption{Wytyczne dla ataku astmy}
\label{fig:ag_ae}
\end{figure}

\begin{figure}[H]
\centering
\includegraphics[scale=0.45]{img/peptic-ulcer.png}
\caption{Wytyczne dla wrzodu trawiennego}
\label{fig:pu}
\end{figure}

Dane opisujące stan pacjenta (odpowiedzi na pytania z wytycznych) są następujące:
\begin{enumerate}
\item{Atak astmy:
	\begin{itemize}
	\item{Respiratory arrest?: no (d\_arrest?no)}
	\item{Response to treatment?: good (d\_response?good)}
	\end{itemize}
}
\item{Wrzód trawienny:
	\begin{itemize}
	\item{Complicated ulcer symptoms?: yes (d\_symptoms?yes)}
	\end{itemize}
}
\end{enumerate}

Dla rozważanych wytycznych możliwe są następujące konflikty:
\begin{enumerate}
\item c\_pe a\_oral\_cortico: replace a\_oral\_cortico with a\_inh\_cortico2
\item \textbf{c\_pe a\_nsaid: add a\_ppi after a\_nsaid}
\item a\_et a\_inh\_cortico: replace a\_inh\_cortico with a\_oral\_cortico2
\end{enumerate}

% SW: W grafach wynikowych warto byłoby zaznaczyć zmodyfikowane węzły (np. innym kolorem tekstu) oraz wspomnieć o tym w opisie
Graf wynikowy dla ataku astmy przedstawiono na rys. \ref{fig:rozw_ae}, natomiast graf wynikowy dla wrzodu trawiennego jest identyczny jak ten z rys. \ref{fig:pu}.
% SW: w tym grafie należałoby zmienić opis węzła PPI -> PPI (a_ppi), aby zachować taką samą konwencję, jak w przypadku pozostałych
\begin{figure}[H]
\centering
\includegraphics[scale=0.45]{img/rozwiazanie1asthma.png}
\caption{Graf wynikowy dla ataku astmy}
\label{fig:rozw_ae}
\end{figure}
\newpage
\section{Przykład 2 - migotanie przedsionków, przewlekła choroba nerek i nadciśnienie}
Wytyczne dla migotania przedsionków (ang. \textit{atrial fibrillation}) przedstawiono na rys. \ref{fig:afib}, dla dla przewlekłej choroby nerek (ang. \textit{chronic kidney disease}) na rys. \ref{fig:ckd}, a dla dla nadciśnienia (ang. \textit{hypertension}) na rys. \ref{fig:htn}.
% SW: tutaj zmiemiłbym identyfikator węzła decyzyjnego: d_paroxys_afib -> d_afib (większa spójność z innymi grafami). Podobna zmiana powinna zostać też wprowadzona do tekstu.

\begin{figure}[H]
\centering
\includegraphics[scale=0.45]{img/afib-ver-4.png}
\caption{Wytyczne dla migotania przedsionków}
\label{fig:afib}
\end{figure}


\begin{figure}[H]
\centering
\includegraphics[scale=0.4]{img/ckd-simplified-ver-5.png}
\caption{Wytyczne dla przewlekłej choroby nerek}
\label{fig:ckd}
\end{figure}

% SW: tutaj zmieniłbym identyfikatory niektórych wierzchołków w grafie na bardziej spójne z tym, co widzieliśmy w poprzednich grafach:
% d_age_under_55 -> d_age
% d_bp_controlled_1 -> d_bp_1
% d_bp_controlled_2 -> d_bp_2
% d_bp_controlled_3 -> d_bp_3

\begin{figure}[H]
\centering
\includegraphics[scale=0.45]{img/htn-ver-3.png}
\caption{Wytyczne dla nadciśnienia}
\label{fig:htn}
\end{figure}

Dane opisujące stan pacjenta (odpowiedzi na pytania z wytycznych) są następujące:
\begin{enumerate}
\item{Migotanie przedsionków:
	\begin{itemize}
	\item{AFib type?: paroxysmal (d\_afib?paroxysmal)}
	\end{itemize}
}
\item{Przewlekła choroba nerek:
	\begin{itemize}
	\item{GFR level?: >=40 (d\_gfr?>=40)}
	\end{itemize}
}
\item{Nadciśnienie:
	\begin{itemize}
	\item{Age?: >=55 (d\_age?>=55)}
	\item{BP Test \#1?: controlled (d\_bp\_1?controlled)}
	\end{itemize}
}
\item Dodatkowe dane, które nie pojawiły się jawnie w wytycznych i które zostały uzupełnione podczas fazy poszukiwania konfliktów:
	\begin{itemize}
	\item{CHA2DS2-VASc = 5 (\&CHA2DS2-VASc?5)}
	\end{itemize}
\end{enumerate}

Dla rozważanych wytycznych możliwe są następujące konflikty:
\begin{enumerate}
\item \textbf{c\_htn c\_ckd: remove a\_step1\_acei, remove a\_step1\_ccb}
\item \textbf{c\_afib c\_ckd c\_htn: remove a\_step3\_diuretric}
\item \textbf{c\_afib c\_ckd: replace a\_antiplatelets with a\_warfarin, replace a\_rct\_a with a\_bb}
\item \textbf{c\_afib c\_ckd \&CHA2DS2-VASc>2: replace a\_mt\_asa with a\_warfarin\_2}
\item c\_afib c\_ckd \&CHA2DS2-VASc<=1: replace a\_oa\_w with a\_aspirin\_1, replace a\_ltoa\_w with a\_aspirin\_2
\end{enumerate}

Graf wynikowy dla migotania przedsionków przedstawiono na rys. \ref{fig:rozw_afib}, dla przewlekłej choroby nerek na rys. \ref{fig:rozw_ckd}, natomiast dla nadciśnienia na rys. \ref{fig:rozw_htn}.

% SW: W tym grafie powinien zmienić Pan opis węzła BB -> BB (a_bb)
\begin{figure}[H]
\centering
\includegraphics[scale=0.45]{img/rozwiazanie1afib-ver-4.png}
\caption{Graf wynikowy dla migotania przedsionków}
\label{fig:rozw_afib}
\end{figure}

% SW: w tym grafie zmieniamy Warfarin -> Warfarin (a_warfarin) oraz Maintenance therapy: warfarin -> Maintenance therapy: warfarin (a_warfarin_2)
\begin{figure}[H]
\centering
\includegraphics[scale=0.4]{img/rozwiazanie1ckd-simplified-ver-5.png}
\caption{Graf wynikowy dla przewlekłej choroby nerek}
\label{fig:rozw_ckd}
\end{figure}
\begin{figure}[H]
\centering
\includegraphics[scale=0.45]{img/rozwiazanie1htn-ver-3.png}
\caption{Graf wynikowy dla nadciśnienia}
\label{fig:rozw_htn}
\end{figure}


\newpage
\section{Przykład 3 - wrzód dwunastnicy i przemijający atak niedokrwienny}
Wytyczne dla wrzodu dwunastnicy (ang. \textit{duodenal ulcer}) przedstawiono na rys. \ref{fig:du}, natomiast dla przemijającego ataku niedokrwiennego (ang. \textit{transient ischemic attack}) na rys. \ref{fig:tia}.
\begin{figure}[H]
\centering
\includegraphics[scale=0.45]{img/du.png}
\caption{Wytyczne dla wrzodu dwunastnicy}
\label{fig:du}
\end{figure}


\begin{figure}[H]
\centering
\includegraphics[scale=0.45]{img/tia.png}
\caption{Wytyczne dla przemijającego ataku niedokrwiennego}
\label{fig:tia}
\end{figure}

Dane opisujące stan pacjenta (odpowiedzi na pytania z wytycznych) są następujące:
\begin{enumerate}
\item{Wrzód dwunastnicy:
	\begin{itemize}
	\item{H.pylori?: negative (d\_hyplori?negative)}
	\item{Z-E syndrome?: positive (d\_ze\_syndrome?positive)}
	\end{itemize}
}
\item{Przemijający atak niedokrwienny:
	\begin{itemize}
	\item{Hypoglycemia?: negative (d\_hypoglycemia?negative)}
	\item{Face-arm-speech test (FAST)?: positive (d\_fast?positive)}
	\item{Neurological symptoms?: resolved (d\_neuro\_symptoms?resolved)}
	\item{Risk of stroke?: elevated (d\_stroke\_risk?elevated)}
	\end{itemize}
}
\end{enumerate}
Dla rozważanych wytycznych możliwe są następujące konflikty:
\begin{enumerate}
\item c\_du a\_aspirin not(a\_ppi) not(a\_dipyridamole): replace a\_aspirin with a\_cl
\item \textbf{c\_du a\_aspirin not(a\_ppi) a\_dipyridamole: add a\_ppi\_2 after d\_ze\_synd\-rome?positive, decrease\_dosage a\_aspirin 50}
\end{enumerate}

Graf wynikowy dla wrzodu dwunastnicy przedstawiono na rys. \ref{fig:rozw_du}, natomiast dla przemijającego ataku niedokrwiennego na rys. \ref{fig:rozw_tia}.

% SW: W tym grafie proszę zmienić Proton-pump inhibitors (PPI) -> Proton-pump inhibitors (a_ppi_2)
\begin{figure}[H]
\centering
\includegraphics[scale=0.5]{img/rozwiazanie1du.png}
\caption{Graf wynikowy dla wrzodu dwunastnicy}
\label{fig:rozw_du}
\end{figure}
\begin{figure}[H]
\centering
\includegraphics[scale=0.5]{img/rozwiazanie1tia.png}
\caption{Graf wynikowy dla przemijającego ataku niedokrwiennego}
\label{fig:rozw_tia}
\end{figure}


\chapter{Podsumowanie}

\section{Osiągnięte cele}
% SW: Tutaj warto jeszcze raz krótko opisać, jakie cele postawiliśmy sobie przy realizacji tej pracy (uogólnienie podejścia bazującego na CLP oraz implementacja rozszerzonego podejścia w formie interaktywnego SWDK).

Podsumowując, wszystkie zamierzenia pracy zostały zrealizowane. Wynikiem pracy jest działający program wyszukujący konflikty występujące między stosowanymi terapiami chorób i proponujący rozwiązania ewentualnych konfliktów. 

\section{Problemy przy realizacji pracy}
% SW: Czy konieczność opanowania nowej metodologii (CLP) oraz powiązanych narzędzi nie stanowiła również pewnego problemu? Jeśli tak, to warto o tym wspomnieć w tym miejscu, podkreślając, że udało się tego Panu dokonać.
Do problemów przy realizacji pracy należy zaliczyć kwestię związaną z wyborem biblioteki służącej do przetwarzania grafów. Ostatecznie wybrana została biblioteka JPGD, ponieważ jest to dość prosta biblioteka. W bardzo łatwy sposób uzyskuje się dostęp do obiektu klasy Graph i podrzędnych obiektów klas Node oraz Edge. Niestety, skorzystanie z tej biblioteki wiązało się z naprawą pewnych błędów związanych z ponownym uzyskiwaniem grafu w wersji tekstowej. Konieczna była modyfikacja funkcji toString dla klas Graph, Node oraz Edge, ponieważ generowane początkowo przez bibliotekę grafy w wersji tekstowej nie pozwalały na wygenerowanie grafu w wersji obrazkowej przez program dot.exe. Przyczyna błędu tkwiła w tym, że biblioteka nie radziła sobie z pustymi wartościami atrybutów. Ponadto, trzeba było zrezygnować z korzystania z podgrafów, ponieważ były one niewłaściwie przez bibliotekę interpretowane. 

Kolejnym problemem przy realizacji pracy była konieczność zapoznania się z bibliotekami Choco, JPGD oraz oprogramowaniem Graphviz. W przypadku bibliotek Choco i JPGD konieczne było zrozumienie ich dokumentacji. Jeśli chodzi o Graphviz, to główny problem stanowiło zapoznanie się z działaniem programów do tworzenia grafów w formie konsolowej (dot.exe) oraz okienkowej (gvedit.exe).

\section{Kierunki dalszego rozwoju}

% SW: Kierunki dlaszego rozwoju można podzielić na te związane z samym podejściem (metodą) oraz na te związane z programem (implementacją). W ramach tych pierwszych można wspomnieć o uwzględnianiu czasu w wytycznych oraz konfliktach (np. konflikty zachodzą, jeśli dwie akcje wykonywane są w tym samym oknie czasowym -- w przeciwnym razie konfliktu nie ma) oraz o uwzględnianiu kosztów podczas usuwania konfliktów (jeśli kilka sposobów usunięcia konfliktu jest możliwych, wówczas wybieramy ten "najtańszy"). Jeśli chodzi o natomiast o implementację, to tutaj można rozważyć inne metody interakcji (w tym większą "klikalność"), połączenie z zewnętrznymi systemami w celu importu danych pacjenta oraz przygotowanie wersji mobilnej.

Kierunki dalszego rozwoju dzielą się na kierunki związane z podejściem oraz związane z implementacją. Do tych pierwszych można zaliczyć uwzględnianie czasu w wytycznych i konfliktach, które polegałoby na tym, że konflikt występuje wtedy, gdy dwie akcje są wykonywane w tym samym czasie. Ponadto, do kierunków związanych z podejściem mogłoby należeć uwzględnianie kosztów w konfliktach. Wtedy metoda wybierałaby rozwiązanie konfliktu o najmniejszym koszcie. 

Jeśli chodzi o implementację, to dalszy rozwój projektu mógłby dążyć do bardziej interaktywnej odpowiedzi na pytania. Program mógłby pozwalać na klikanie na krawędzie grafu zamiast wybierać odpowiedzi za pomocą 
% SW: Termin "radiobutton" ma charakter slangowy -- lepiej mówić o "polu wyboru".
pól wyboru. Ponadto, program mógłby wspierać także inne formaty grafów, nie tylko format Graphviza o rozszerzeniu dot. Dobrymi pomysłami byłyby także integracja programu z zewnętrznymi systemami w celu pobrania danych pacjenta oraz przygotowanie wersji na aplikacje mobilne.




\backmatter
\addcontentsline{toc}{chapter}{Bibliografia}
% rodzaj bibliografii
\bibliographystyle{plain}
% plik z wpisami bibliograficznymi
\bibliography{bibliografia}

\end{spacing}
\end{document}