\chapter{Podsumowanie}

\section{Osiągnięte cele}
% SW: Tutaj warto jeszcze raz krótko opisać, jakie cele postawiliśmy sobie przy realizacji tej pracy (uogólnienie podejścia bazującego na CLP oraz implementacja rozszerzonego podejścia w formie interaktywnego SWDK).

Podsumowując, wszystkie zamierzenia pracy zostały zrealizowane. Wynikiem pracy jest działający program wyszukujący konflikty występujące między stosowanymi terapiami chorób i proponujący rozwiązania ewentualnych konfliktów. 

\section{Problemy przy realizacji pracy}
% SW: Czy konieczność opanowania nowej metodologii (CLP) oraz powiązanych narzędzi nie stanowiła również pewnego problemu? Jeśli tak, to warto o tym wspomnieć w tym miejscu, podkreślając, że udało się tego Panu dokonać.
Do problemów przy realizacji pracy należy zaliczyć kwestię związaną z wyborem biblioteki służącej do przetwarzania grafów. Ostatecznie wybrana została biblioteka JPGD, ponieważ jest to dość prosta biblioteka. W bardzo łatwy sposób uzyskuje się dostęp do obiektu klasy Graph i podrzędnych obiektów klas Node oraz Edge. Niestety, skorzystanie z tej biblioteki wiązało się z naprawą pewnych błędów związanych z ponownym uzyskiwaniem grafu w wersji tekstowej. Konieczna była modyfikacja funkcji toString dla klas Graph, Node oraz Edge, ponieważ generowane początkowo przez bibliotekę grafy w wersji tekstowej nie pozwalały na wygenerowanie grafu w wersji obrazkowej przez program dot.exe. Przyczyna błędu tkwiła w tym, że biblioteka nie radziła sobie z pustymi wartościami atrybutów. Ponadto, trzeba było zrezygnować z korzystania z podgrafów, ponieważ były one niewłaściwie przez bibliotekę interpretowane.

\section{Kierunki dalszego rozwoju}

% SW: Kierunki dlaszego rozwoju można podzielić na te związane z samym podejściem (metodą) oraz na te związane z programem (implementacją). W ramach tych pierwszych można wspomnieć o uwzględnianiu czasu w wytycznych oraz konfliktach (np. konflikty zachodzą, jeśli dwie akcje wykonywane są w tym samym oknie czasowym -- w przeciwnym razie konfliktu nie ma) oraz o uwzględnianiu kosztów podczas usuwania konfliktów (jeśli kilka sposobów usunięcia konfliktu jest możliwych, wówczas wybieramy ten "najtańszy"). Jeśli chodzi o natomiast o implementację, to tutaj można rozważyć inne metody interakcji (w tym większą "klikalność"), połączenie z zewnętrznymi systemami w celu importu danych pacjenta oraz przygotowanie wersji mobilnej.

Dalszy rozwój projektu mógłby dążyć do bardziej interaktywnej odpowiedzi na pytania. Program mógłby pozwalać na klikanie na krawędzie grafu 
zamiast wybierać odpowiedzi za pomocą 

% SW: Termin "radiobutton" ma charakter slangowy -- lepiej mówić o "polu wyboru".

radiobuttonów. Ponadto, program mógłby wspierać także inne formaty grafów, nie tylko format Graphviza o rozszerzeniu dot. 
