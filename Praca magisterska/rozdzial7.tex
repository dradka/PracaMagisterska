\chapter{Podsumowanie}

\section{Osiągnięte cele}
W pracy udało się zrealizować wszystkie postawione cele. W szczególności rozszerzono podejście oparte na programowaniu logicznym z ograniczeniami zaproponowane w \cite{SzWilk2} pozwalając na stosowanie więcej niż dwóch wytycznych, dopuszczając złożone zmiany w wytycznych obejmujące wiele operacji oraz uwzględniając dawki podawanych leków. Rozszerzone podejście zaimplementowane zostało w formie interaktywnego systemu wspomagania decyzji klinicznych, który wyszukuje konflikty między zastosowanymi wytycznymi, wprowadza niezbędne zmiany do wytycznych i proponuje bezpieczne (tzn. pozbawione konfliktów) terapie.

\section{Problemy przy realizacji pracy}
Do problemów przy realizacji pracy należy zaliczyć kwestię związaną z wyborem biblioteki służącej do przetwarzania grafów. Ostatecznie wybrana została biblioteka JPGD z uwagi na jej prostotę. W bardzo łatwy sposób uzyskuje się dostęp do obiektu klasy \texttt{Graph} i podrzędnych obiektów klas \texttt{Node} oraz \texttt{Edge}. Niestety, skorzystanie z tej biblioteki wiązało się z naprawą błędów występujących podczas tworzenia tekstowej reprezentacji grafu. W szczególności konieczna była modyfikacja metody \texttt{toString} w klasach \texttt{Graph}, \texttt{Node} oraz \texttt{Edge}, ponieważ generowane początkowo przez bibliotekę tekstowe reprezentacje grafów nie były poprawnie przetwarzane przez program \texttt{dot}. Przyczyna błędu tkwiła w tym, że biblioteka JGPD nie radziła sobie z pustymi wartościami atrybutów wspomnianych obiektów. Ponadto trzeba było zrezygnować z korzystania z podgrafów, ponieważ były one niewłaściwie przez bibliotekę interpretowane. 

Kolejnym problemem przy realizacji pracy była konieczność zapoznania się z bibliotekami Choco, JPGD oraz oprogramowaniem Graphviz. W przypadku bibliotek Choco i JPGD konieczne było zrozumienie ich dokumentacji. Jeśli chodzi o Graphviz, to główny problem stanowiło zapoznanie się z działaniem programów do tworzenia grafów w formie konsolowej (\texttt{dot}) oraz okienkowej (\texttt{gvedit}).

\section{Kierunki dalszego rozwoju}

% SW: Kierunki dlaszego rozwoju można podzielić na te związane z samym podejściem (metodą) oraz na te związane z programem (implementacją). W ramach tych pierwszych można wspomnieć o uwzględnianiu czasu w wytycznych oraz konfliktach (np. konflikty zachodzą, jeśli dwie akcje wykonywane są w tym samym oknie czasowym -- w przeciwnym razie konfliktu nie ma) oraz o uwzględnianiu kosztów podczas usuwania konfliktów (jeśli kilka sposobów usunięcia konfliktu jest możliwych, wówczas wybieramy ten "najtańszy"). Jeśli chodzi o natomiast o implementację, to tutaj można rozważyć inne metody interakcji (w tym większą "klikalność"), połączenie z zewnętrznymi systemami w celu importu danych pacjenta oraz przygotowanie wersji mobilnej.

Kierunki dalszego rozwoju związane są zarówno z podejściem teoretycznym, jak i jego implementacją w formie systemu wspomagania decyzji. W ramach pierwszej grupy można rozważyć uwzględnianie czasu w wytycznych i konfliktach -- konflikty występowałyby tylko wtedy, gdy dwie akcje byłyby wykonywane w tym samym czasie. Ponadto można byłoby uwzględnić koszty zmian związanych z usuwaniem konfliktu. Wtedy metoda wybierałaby rozwiązanie konfliktu o najmniejszym koszcie. 

Jeśli chodzi o zaimplementowany system, to jego dalszy rozwój mógłby obejmować zastosowanie technik manipulacji bezpośredniej podczas udzielania odpowiedzi na pytania -- program mógłby pozwalać na wybieranie krawędzi grafu zamiast pól wyboru. Ponadto program mógłby wspierać także inne reprezentacje grafów, nie tylko DOT. Dobrymi pomysłami byłyby także integracja programu z zewnętrznymi systemami w celu pobrania danych pacjenta oraz przygotowanie wersji na platformy mobilne.


