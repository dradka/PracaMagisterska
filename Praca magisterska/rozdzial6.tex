\chapter{Działanie systemu}
Działanie systemu zostanie zaprezentowane w oparciu o kilka przykładów. Dla każdego przykładu przedstawione zostały grafy z zaznaczonymi ścieżkami, które zostały wybrane podczas etapu udzielania odpowiedzi. Odpowiedzi na pytania są także przedstawione w formie tekstowej. Następnie, dla każdego przykładu przedstawiono listę konfliktów, które mogą wystąpić oraz zmiany, jakie należy wprowadzić w przypadku ich wystąpienia. Na końcu przykładów zaprezentowane są znalezione konflikty.
% SW: W obecnej formie ta sekcja jest bardzo skrótowa. Powinien Pan w "rozbiegówce" wyjaśnić, że pokaże Pan działanie systemu na kilku przykładach. Większość to rzeczywiste wytyczne (uproszczone na potrzeby testowania. Wykorzystywane są też przykłady sztuczne demonstrujące zachowanie algorytmu w granicznych wypadkach (tutaj chodzi mi o przykład, gdzie w ramach usuwania jednego konfliktu wprowadzanych jest kilka zmian). Jeśli natomiast chodzi o przykłady rzeczywiste, to dodałbym także przykład z dawkami leków -- brakuje go do pełnej demonstracji Pana dokonań.

% SW: Również na wstępie powinien Pan wyjaśnić, że opisując poszczególne przykłady pokaże Pan rozważane wytyczne w formie grafów z zaznaczonymi ścieżkami odpowiadającymi dostępnym danym pacjenta, dostępną wiedzę na temat konfliktów i sposobów ich rozwiązania, a także wynikowe grafy (a przynajmniej ich zmodyfikowane fragmenty). Przedstawiając konflikty powinien zaznaczyć Pan te, które zostały wykryte i usunięte dla danego pacjenta. Wreszcie konflikty i sposoby ich usunięcia powinien Pan przedstawić korzystając ze składni, którą Pan zaproponował (jest to bardzo eleganckie rozwiązanie i warto się nim pochwalić).

\section{Przypadek 1 - atak astmy i wrzód trawienny}
Wytyczne dla ataku astmy (ang. \textit{asthma exacerbation}) przedstawiono na rys. \ref{fig:ag_ae}.\\*
Wytyczne dla wrzodu trawiennego (ang. \textit{peptic ulcer}) przedstawiono na rys. \ref{fig:pu}.

% SW: Każdy graf, zestaw danych pacjenta oraz lista konfliktów powinien zostac potraktowany jako rysunek z numerem i tytułem (wprowadziłem odpowiednią zmianę poniżej), a w tekście powinno się do niego znaleźć odwołanie).

% SW: Warto byłoby ujednolicić wszystkie diagramy, aby w wierzchołku kontekstowym pojawiał sie zawsze akronim lub nazwa choroby. Obecnie mamy albo choroby, albo zdania "Patient diagnosed with ...".  
\begin{figure}[H]
\centering
\includegraphics[width=0.5\textwidth]{img/asthma.png}
\caption{Wytyczne dla ataku astmy}
\label{fig:ag_ae}
\end{figure}

\newpage
\begin{figure}[H]
\centering
\includegraphics[width=0.7\textwidth]{img/peptic-ulcer.png}
\caption{Wytyczne dla wrzodu trawiennego}
\label{fig:pu}
\end{figure}
\newpage
\noindent Odpowiedzi na pytania:
\begin{enumerate}
\item{Atak astmy:
	\begin{itemize}
	\item{Respiratory arrest?: no}
	\item{Response to treatment?: good}
	\end{itemize}
}
\item{Wrzód trawienny:
	\begin{itemize}
	\item{Complicated ulcer symptoms?: yes}
	\end{itemize}
}
\end{enumerate}
Konflikty:
\begin{verbatim}
If diagnosed(PE) and execute(oral corticosteroids), then 
replace execute(oral corticosteroids) -> execute(inhaled corticosteroids)

If diagnosed(PE) and execute(NSAID)
then add execute(PPI) after execute(NSAID)

If execute(eradication therapy) and execute(inhaled corticosteroids), then
replace execute(inhaled corticosteroids) -> execute(oral corticosteroids)
\end{verbatim}
Znalezione konflikty:
\begin{verbatim}
If diagnosed(PE) and execute(NSAID) 
then add execute(PPI) after execute(NSAID)
\end{verbatim}
\newpage
\section{Przypadek 2 - migotanie przedsionków, przewlekła choroba nerek i nadciśnienie}
Wytyczne dla migotania przedsionków (ang. \textit{atrial fibrillation}) przedstawiono na rys. \ref{fig:afib}.\\*
Wytyczne dla przewlekłej choroby nerek (ang. \textit{chronic kidney disease}) przedstawiono na rys. \ref{fig:htn}.\\*
Wytyczne dla nadciśnienia (ang. \textit{hypertension}) przedstawiono na rys. \ref{fig:htn}.
\begin{figure}[H]
\centering
\includegraphics[width=0.5\textwidth]{img/afib-ver-4.png}
\caption{Wytyczne dla migotania przedsionków}
\label{fig:afib}
\end{figure}
\newpage
\begin{figure}[H]
\centering
\includegraphics[width=0.8\textwidth]{img/ckd-simplified-ver-5.png}
\caption{Wytyczne dla przewlekłej choroby nerek}
\label{fig:ckd}
\end{figure}
\newpage
\begin{figure}[H]
\centering
\includegraphics[width=0.7\textwidth]{img/htn-ver-3.png}
\caption{Wytyczne dla nadciśnienia}
\label{fig:htn}
\end{figure}
\newpage
\noindent Odpowiedzi na pytania:
\begin{enumerate}
\item{Migotanie przedsionków:
	\begin{itemize}
	\item{AFib type?: paroxysmal}
	\end{itemize}
}
\item{Przewlekła choroba nerek:
	\begin{itemize}
	\item{GFR level?: >=40}
	\end{itemize}
}
\item{Nadciśnienie:
	\begin{itemize}
	\item{Age?: >=55}
	\item{BP Test \#1?: controlled}
	\end{itemize}
}
% SW: Tutaj warto zaznaczyć, że te dane (to pytanie) nie pojawiają się jawnie w żadnych rozważanych wytycznych, ale są uwzględnione podczas sprawdzania konfliktów, dlatego użytkownik jest proszony o podanie odpowiedniej wartości.
\item{CHA2DS2-VASc = 5 (Parametr ten nie pojawia się jawnie w wytycznych, użytkownik jest proszony o podanie tej wartości)}
\end{enumerate}
Konflikty:
\begin{verbatim}
If patient diagnosed with HTN and CKD, then remove Step 1 from the algorithm for HTN

If patient diagnosed with HTN, AFib and CKD, then delete “diuretics” from Step 3 
in the algorithm for HTN

If patient diagnosed with CKD and AFib, then replace aspirin with warfarin 
in the algorithm for CKD

If patient diagnosed with AFib and CKD, then replace amiodarone with BB 
in the algorithm for AFib

If patient diagnosed with AFib and CKD, and CHA2DS2-VASc > 2
then replace aspirin with warfarin as maintenance therapy in the algorithm for AFib

If patient diagnosed with Afib and CKD, and CHA2DS2-VASc <= 1
then replace warfarin with aspirin as the long-term therapy in the algorithm for AFib
\end{verbatim}
Znalezione konflikty:
\begin{verbatim}
If patient diagnosed with HTN and CKD, then remove Step 1 from the algorithm for HTN

If patient diagnosed with HTN, AFib and CKD, then delete “diuretics” from Step 3
in the algorithm for HTN

If patient diagnosed with CKD and AFib, then replace aspirin with warfarin 
in the algorithm for CKD

If patient diagnosed with AFib and CKD, then replace amiodarone with BB 
in the algorithm for AFib

If patient diagnosed with AFib and CKD, and CHA2DS2-VASc > 2
then replace aspirin with warfarin as maintenance therapy in the algorithm for AFib
\end{verbatim}
\newpage
\section{Przypadek 3 - wrzód dwunastnicy i przemijający atak niedokrwienny}
Wytyczne dla wrzodu dwunastnicy (ang. \textit{duodenal ulcer}) przedstawiono na rys. \ref{fig:du}.\\*
Wytyczne dla przemijającego ataku niedokrwiennego (ang. \textit{transient ischemic attack}) przedstawiono na rys. \ref{fig:tia}.
\begin{figure}[H]
\centering
\includegraphics[width=0.7\textwidth]{img/du.png}
\caption{Wytyczne dla wrzodu dwunastnicy}
\label{fig:du}
\end{figure}
\newpage
\begin{figure}[H]
\centering
\includegraphics[width=0.7\textwidth]{img/tia.png}
\caption{Wytyczne dla przemijającego ataku niedokrwiennego}
\label{fig:tia}
\end{figure}
\newpage
\noindent Odpowiedzi na pytania:
\begin{enumerate}
\item{Wrzód dwunastnicy:
	\begin{itemize}
	\item{H.pylori?: negative}
	\item{Z-E syndrome?: positive}
	\end{itemize}
}
\item{Przemijający atak niedokrwienny:
	\begin{itemize}
	\item{Hypoglycemia?: negative}
	\item{Face-arm-speech test (FAST)?: positive}
	\item{Neurological symptoms?: resolved}
	\item{Risk of stroke?: elevated}
	\end{itemize}
}
\end{enumerate}
Konflikty:
\begin{verbatim}
if diagnosed(DU) and execute(A) and not execute(PPI) and not execute(D), then 
(1) replace execute(A) -> execute(CL)

if diagnosed(DU) and execute(A) and not execute(PPI) and execute(D), then
(1) replace not execute(PPI) -> execute(PPI) (wprowadź PPI)
(2) replace dosage(A, x) -> dosage(A, x – 50) (obniż dawkę A o 50 jednostek)
\end{verbatim}
Znalezione konflikty:
\begin{verbatim}
if diagnosed(DU) and execute(A) and not execute(PPI) and execute(D), then
(1) replace not execute(PPI) -> execute(PPI) (wprowadź PPI)
(2) replace dosage(A, x) -> dosage(A, x – 50) (obniż dawkę A o 50 jednostek)
\end{verbatim}
