\chapter{Programowanie logiczne z ograniczeniami (CLP)}

% SW: Powiedziałbym, że CLP pozwala na użycie programowania logicznego do rozwwiązywania problemów z ograniczeniami (constraint satisfaction problems, CSP). 
Programowanie logiczne z ograniczeniami\cite{CLP} pozwala na użycie programowania logicznego do rozwiązywania problemów z ograniczeniami. Przykładowym ograniczeniem może być wyrażenie postaci X+Y>5. 
% SW: Tutaj raczej należałoby powiedzieć, że program CLP składa się z następujących elementów
Program CLP składa się z następujących elementów:
\begin{itemize}
% SW: Zmienne nie muszą być całkowitoliczbowe -- musza mieć jednak skończone dziedziny.
\item{Skończonego zbioru zmiennych z wartościami ze skończonych dziedzin}
\item{Zbioru ograniczeń między zmiennymi}
\item{Rozwiązań problemu polegających na przypisaniu wartości do zmiennych, które spełniają ograniczenia}
\end{itemize}
Przykładowym zastosowaniem programowania logicznego z ograniczeniami jest zagadka SEND + MORE  = MONEY.\cite{Eclipse} Zagadka ta polega na przypisaniu cyfr z zakresu od 0 do 9 do zmiennych odpowiadających literom zawartym w równaniu tak, aby równanie było spełnione. Każda litera ma swoją unikalną wartość cyfrową. Ponadto litery S i M mają wartości różne od 0.

% SW: Każdy z takich przykładów należałoby potraktować jako osobny rysunek i zamieścić do nich odwołanie w tekście.

\begin{verbatim}
    SEND
  + MORE
 -------
   MONEY
\end{verbatim}
% SW: Tutaj warto wyjaśnić, że mówimy o środowisku do uruchamiania programów CLP, a nie o popularnym środowisku programistycznym. Poza tym warto dodać, że nieco bardziej szczegółowy opis środowiska znajduje się w punkcie 4.1.
Rozwiązaniem tego problemu jest następujący program napisany w ECLiPSe (jest to kompilator programów CLP, a nie popularne środowisko programistyczne Eclipse, bardziej szczegółowy opis tego programu znajduje się w punkcie 4.1):
\newpage
\begin{verbatim}
:-lib(ic).
sendmore1(Digits):-
Digits = [S,E,N,D,M,O,R,Y],
Digits :: [0..9],
alldifferent(Digits),
S #\= 0,
M #\= 0,
1000*S + 100*E + 10*N + D
+ 1000*M + 100*O + 10*R + E
#= 10000*M + 1000*O + 100*N + 10*E + Y,
labeling(Digits).
\end{verbatim}
Po skompilowaniu tego programu wystarczy wywołać funkcję sendmore1(Digits), aby otrzymać rozwiązanie zagadki. Rozwiązaniem jest następujące przypisanie cyfr do zmiennych: S=9, E=5, N=6, D=7, M=1, O=0, R=8, Y=2. 

Można zauważyć na podstawie przykładu, że komendy w ECLiPSe składają się ze zdań zakończonych kropką, poszczególne fragmenty zdań są oddzielone od siebie przecinkami. Znak równości między zmiennymi lub wartościami liczbowymi to „\#=”, znak nierówności to „\#\textbackslash=”. Można stosować także operatory and i or i przypisywać ich wartość do zmiennych za pomocą znaku równości. 

CLP może być wykorzystane do rozwiązywania innych zadań.\cite{niederlinski} Do przykładowych zadań należy popularne zadanie z farmerem, wilkiem, gęsią i kapustą. Polega ono na tym, że farmer posiada łódkę, na której może się zmieścić on i jeszcze jeden element. Zadaniem farmera jest przewieźć wszystkie elementy z jednej strony rzeki na drugą. Problem stanowi fakt, że farmer nie może zostawić wilka i gęsi samych, bo wilk zje gęś, a także nie może zostawić gęsi z kapustą, ponieważ gęś zje kapustę. Innym przykładem jest zadanie o nazwie osiem królowych. Polega ono na tym, że na szachownicy o wymiarach 8x8 należy umieścić 8 królowych w taki sposób, aby żadna z nich nie mogła zbić innej królowej.

Programowanie logiczne z ograniczeniami może być również użyte do rozwiązywania trudniejszych problemów. Do problemów tych należą m.in. harmonogramowanie pracy lakierni samochodowej czy projektowanie inteligentnych systemów okablowania dla dużych budynków.
% SW: Warto podać rozwiązanie tej zagadki i przypisanie cyfr do poszczególnych zmiennych. Poza tym warto też krótko wyjaśnić, że CLP może być stosownae do znacznie poważniejszych problemów i podać ich przykłady. Są one wymienione w książce Niederlińskiego (do pobrania za darmo ze strony ECLiPSe'a) na stronach 7 i 8.
