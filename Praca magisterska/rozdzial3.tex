\chapter{Programowanie logiczne z ograniczeniami}

\label{sec:clp}

Programowanie logiczne z ograniczeniami \cite{CLP} wykorzystuje programowania logicznego (ang. \textit{logic programming}, LP) do rozwiązywania problemów spełaniania ograniczeń (ang. \textit{constraint satisfaction problem}, CSP). 

Program CLP składa się z następujących elementów:
\begin{itemize}
\item{zbioru zmiennych o wartościach ze skończonych dziedzin,}
\item{zbioru ograniczeń narzuconych na zmienne (np. X + Y > 5).}
\end{itemize}

Rozwiązaniem programu CLP jest takie przyporządkowanie wartości do zmiennych, aby spełnione zostały wszystkie zdefiniowane ograniczenia.

Poniżej opisano zastosowanie CLP do rozwiązania zagadki logicznej SEND + MORE  = MONEY \cite{Eclipse} pokazanej na rys. \ref{fig:sendmoremoney}. Rozwiązaniem tej zagadki jest takie przypisaniu cyfr z zakresu od 0 do 9 do zmiennych odpowiadających literom zawartym w równaniu, aby było ono spełnione. Każda zmienna (litera) powinna mieć unikalną wartość. Ponadto wartości zmiennych S i M muszą być różne od 0.

\begin{figure}
\begin{center}
\begin{verbatim}
    SEND
  + MORE
 -------
   MONEY
\end{verbatim}
\end{center}
\caption{Zagadka SEND + MORE = MONEY}
\label{fig:sendmoremoney}
\end{figure}
% SW: Tutaj warto wyjaśnić, że mówimy o środowisku do uruchamiania programów CLP, a nie o popularnym środowisku programistycznym. Poza tym warto dodać, że nieco bardziej szczegółowy opis środowiska znajduje się w punkcie 4.1.

Do rozwiązania tej zagadki służy następujący program pokazany na rys. \ref{fig:eclipse_program} i przygotowany w środowisku w ECLiPSe (chodzi tutaj o specjalizowane narzędzie dla CLP, a nie popularne środowisko programistyczne dla języka Java -- bardziej szczegółowy opis tego środowiska znajduje się w rozdziale \ref{sec:eclipse}):

\begin{figure}
\begin{verbatim}
:-lib(ic).
sendmore(Digits):-
Digits = [S,E,N,D,M,O,R,Y],
Digits :: [0..9],
alldifferent(Digits),
S #\= 0,
M #\= 0,
1000*S + 100*E + 10*N + D
+ 1000*M + 100*O + 10*R + E
#= 10000*M + 1000*O + 100*N + 10*E + Y,
labeling(Digits).
\end{verbatim}
\caption{Program w środowisku ECLiPSe rozwiązujący zagadkę}
\label{fig:eclipse_program}
\end{figure}

Można zauważyć, że poszczególne klauzule programu w ECLiPSe składają się ze zdań zakończonych kropką, poszczególne fragmenty zdań są oddzielone od siebie przecinkami. Znak równości między zmiennymi lub wartościami liczbowymi to „\#=”, znak nierówności to „\#\textbackslash=”. Można stosować także operatory and i or i przypisywać ich wartość do zmiennych za pomocą znaku równości. 

Po skompilowaniu tego programu wystarczy wywołać funkcję \texttt{sendmore(Digits)}, aby otrzymać rozwiązanie zagadki. Rozwiązaniem jest następujące przypisanie cyfr do zmiennych: S=9, E=5, N=6, D=7, M=1, O=0, R=8, Y=2. 

CLP może być wykorzystywane nie tylko do rozwiązywania popularnych zagadek logicznych (np. osiem królowych), ale jest także stosowane w poważnych, rzeczywistych problemach, m.in. m.in. harmonogramowaniu pracy lakierni samochodowej czy projektowaniu inteligentnych systemów okablowania dla dużych budynków \cite{niederlinski}.

